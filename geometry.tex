\chapter*{Геометриуя}

\epigraph{Уравнения --- просто скучная часть математики.
Я пытаюсь смотреть на мир с точки зрения геометрии.}{---Стивен Хокинг (1942---2018)}

Классическая геометрия в двух- или трёхмерном пространстве, безусловно, неисчерпаемый источник для составителей задач.
Но мы хотим головоломок, а это не те задачи, которые бы Евклид включил в свой Том II.
Поэтому вас не будут здесь просить доказать, что $AB=CD$ или что один треугольник конгруэнтен другому.
По счастью, у нас всё ещё имеется огромный выбор из множества очаровательных геометрических задач.
Задача, которую мы разберём для примера, появилась в 1980 году на подготовительном  школьном экзамене, %(Preliminary Scholastic Aptitude Exam), 
где, к стыду составителей из экзаменационного комитета, % по образованию (англ. ETS)(Educational Testing Service), 
утверждённый правильным ответ оказался неверен.
И один смелый (уверенный в себе) ученик, получив результат экзамена, обратился в %ETS
с претензией.
К нашей радости, %( Luckily for us) К счастью? 
правильное решение являет собой  чудесное интуитивное доказательство.
(Хочу заметить, что впоследствии был создана специальная группа, в которой  трудился и ваш автор, для анализа экзаменационных заданий по математике.)


\subsection*{Склеивание пирамид}% (GLUING PIRAMIDS)


Пирамида с квадратным основанием, имеющая рёбра единичной длины, и пирамида с треугольным основанием (тетраэдр), также с рёбрами единичной длины, склеены вместе по треугольным граням.
  Сколько граней имеет полученный многогранник?


\paragraph{Решение:}

Пирамида с квадратным основанием имеет пять граней, тетраэдр -- четыре.
Так как две грани склеены вместе, у получившегося многогранника будет $7=4+5-2$ граней, правильно?
Это, очевидно, была преполагаемая линия рассуждений.
Составителю задачи могло бы прийти в голову, что, в теории, некая пара прилегающих граней, по одной из каждого многогранника, может в процессе склейки оказаться лежащей в одной плоскости.
Таким образом, они сольются в одну грань, что уменьшает наш подсчёт. %???их число
Но, несомненно, такое совпадение должно быть исключено.
Ведь эти два многогранника даже не одинаковой формы.
   На самом деле, так и случается (дважды).
Склеенный многогранник имеет пять граней.
Вы можете вообразить себе такую картину:  две пирамиды с квадратным основанием стоят на столе основаниями вниз и примыкают друг к другу по ребру основания.
Теперь,
соединим вершины пирамид воображаемой линией, отметим, что длина полученного отрезка равна единице, как и все рёбра пирамиды.

Таким образом, между двумя пирамидами с квадратным основанием мы, в сущности, построили правильный тетраэдр.
Две плоскости, каждая из которой содержит по треугольной грани от каждой из двух пирамид, также содержит и грань тетраэдра.
Отсюда следует ответ задачи. %(the result follows)Отсюда результат???


(Если вы затрудняетесь это представить, смотрите картинку ниже)


  Данное доказательство, иногда называемое «палаточным» решением, появилось в 1982 в статье Стивена Янга ( Steven Young  “The Mental Representation of Geometrical Knowledge”, in the Journal of Mathematical Behavior).


                                Рис. Стр 44




Одна из задач, приведённых ниже,  имеет «доказательство без слов» --- достаточно одного рисунка.
Посмотрим, сможете ли вы догадаться, которая.

\subsection*{Окружности в пространстве}%  (CIRCLES IN SPACE)


Возможно ли трёхмерное пространство разбить на окружности?  




\subsection*{Магия кубов}% (MAGIC WITH CUBES)


Можно ли протащить куб сквозь отверстие в кубе меньшего размера?




\subsection*{Красные точки и синие точки}% (RED POINTS AND BLUE POINTS)


На плоскости дано красных n точек и n синих точек, никакие три точки не лежат на одной прямой.
Докажите, что их можно разбить на пары таким образом, что отрезки (line segments), соединяющие каждую красную точку с соответствующей ей синей не пересекаются.



\subsection*{Прямая через две точки}% (LINE THROUGH TWO POINTS)


Пусть X -- конечное множество точек на плоскости, не все они лежат на одной прямой.
Докажите, что существует прямая, проходящая ровно через две точки множества X.




\subsection*{Пары на максимальном расстоянии}% (PAIRS AT MAXIMUM DISTANCE)


И опять, X -- конечное множество точек на плоскости.
Положим,  X содержит n точек и максимальное расстояние между любыми двумя точками равно d.
Докажите, что существует максимум n пар точек из множества X, расстояние между которыми равно d.




\subsection*{Монах на горе}% (MONK ON A MOUNTAIN)


 В понедельник на рассвете монах начинает восхождение на гору Фудзияма и с приходом ночи достигает вершины.
Он проводит ночь на вершине горы и на следующее утро пускается в обратный путь, добираясь до подножия горы на закате солнца.


  Докажите, что существует такой момент, когда монах в одно и тоже время дня был на одной и той же высоте в понедельник и вторник.


\subsection*{Раскраска многогранника}% (PAINTING THE POLYHEDRON)


Пусть P -- многогранник с красными и зелёными гранями, такими, что каждая красная грань окружена зелёными, но  суммарная площадь красных граней превосходит суммарную площадь зелёных.
Докажите, что невозможно вписать сферу в многогранник P.




\subsection*{Круглые тени}% (CIRCULAR SHADOWS)


Два круга являются проекциями некоего трёхмерного тела на две плоскости.
Докажите, что их радиусы равны.




\subsection*{Полоски на плоскости}% (STRIPS IN THE PLANE) 


«Полоска» -- часть плоскости между двумя параллельными прямыми.
Докажите, что нельзя покрыть плоскость множеством полосок, суммарная ширина которых конечна.




\subsection*{Ромбики в шестиугольнике}% (DIAMONDS IN HEXAGON)


Из большой треугольной решётки вырезали правильный шестиугольник и замостили его «ромбиками» (пара треугольников, склееных вместе по одной из сторон).
Имеется три вида ромбиков, в зависимости от их ориентации.
Докажите, что нужно ровно одно и тоже количество ромбиков каждого вида, чтобы замостить шестиугольник.


                                         Рис. Стр 46
\subsection*{Замощение ромба}%   (RHOMBUS TILING)


Давайте попробуем ещё раз, но плитки будут больше и больше будет количество сторон.



  Создайте ( различных ромбов, используя непараллельные стороны правильного 2n-угольника, затем замостите ими 2n-угольник, используя параллельный перенос ромбов.

Докажите, что каждый различный ромб используется только один раз!




\subsection*{Векторы на многограннике}% (VECTORS ON A POLYHEDRON)


Каждой грани многогранника соответствует вектор, перпендикулярный грани, направленный вовне, и имеющий длину, равную площади грани.
Докажите, что сумма таких векторов равна нулю.




\subsection*{Три окружности}% (THREE CIRCLES)


Назовём фокусом  двух  окружностей  точку пересечения двух прямых, каждая из которых является касательной к обеим окружностям, но не проходит между ними.
Таким образом, три окружности различных радиусов (не лежащих в друг друге)
определяют три фокуса.
Докажите, что эти три фокуса лежат на одной прямой.


                              Рис  1.  Стр. 47


\subsection*{Сфера и четырёхугольник}% (SPHERE AND QUADRIATERAL)


Пространственный четырёхугольник касается всеми сторонами сферы.
Докажите, что все точки касания лежат на одной плоскости.




  Последняя задача является небольшой экскурсией в топологию и в бесконечность разных размеров.




\subsection*{Восьмёрки на плоскости}% (FIGURES 8S IN THE PLANE)


Сколько непересекающихся топологических «восьмёрок»  может быть построено на плоскости?


                                Рис 2.  Стр 47
