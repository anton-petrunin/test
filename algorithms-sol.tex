ИНФЕКЦИЯ НА ШАХМАТНОЙ ДОСКЕ (THE INFECTED CHECKBOARD)


   Эта милая задача впервые появилась в советском журнале “Квант” где-то в 1986 году, а затем перекочевала в Венгрию. В случае произвольного расположения начальных клеток процесс имеет название двухмерная bootstrap перколяция. Прекрасный математический анализ этого процесса был сделан Анде Холройдом (ныне профессор Университета Британской Колумбии). (Ander Holroyd, Probability Theory and Related Fields, Vol. 125, No. 2(2003), pp 195-224) Задача, представленная здесь, попала ко мне от Джоэла Спенсера Joel Spenser, NYU) из Нью-Йо́ркского университе́та, который утверждал, что существует “решение в одну строку”! Как вы увидите, это не слишком большое преувеличение.
   Решающие эту задачу читатели, введённые в заблуждение примером с диагональю, часто пытаются доказать, что начально инфицированные клетки должны быть в каждой строке или столбце. Но это далеко не так.  Обратите внимание, что, например, при расположении больных клеток, как показано на рисунке ниже, инфекция распространится на всю доску.
                                      Рис. Стр 83


   Действительно, существует великое множество (бесчисленное количество) способов заразить всю доску n начально инфицированными клетками, но, оказывается, нет способа сделать это с меньшим количеством клеток. Здесь необходим некий магический параметр P, какой?
   Этот параметр -- периметр! Когда клетка заражена, по меньшей мере две из её сторон сливаются с внутренностью инфицированной территории, и максимум две стороны добавляются к границе инфицированной территории. Следовательно, периметр инфицированной территории не может увеличиваться. Поскольку периметр всей доски равен 4n (предполагая единичную длину стороны клетки), начальная инфицированная территория должна содержать минимум n клеток.


   Дополнительное упражнение для тех, кому интересно: докажите, что n начально инфицированных клеток необходимо даже тогда, когда верх и низ доски склеены, чтобы получился цилиндр. Если стороны так же склеены, и получается тор, то достаточно (и необходимо) n-1 начально инфицированных клеток. Периметр здесь больше не работает,
решает проблему другой подход, найденный Брюсом Рихтером (университет Ватерлоо, Уо́терлу) (Bruce Richter, University of Waterloo) и вашим автором.




ПУСТОЕ ВЕДРО (EMPTYING A BUCKET)


   Ещё одна красивая задача из бывшего Советского Союза, которая была представлена на V-й Всесоюзной математической олимпиаде в Риге в 1971 году. Она появлялась позже  (правда, без хозяйственного инвентаря), на Математической олимпиаде Патнема в 1983 году. Задача попала ко мне от Кристиана Боргса из лабораторий “Microsoft Research” (Christian Borgs of Microsoft Research). Я покажу два решения -- одно моё, комбинаторное, и второе, элегантное, теоретико-числовое доказательство, придуманное Сванте Янсоном из Уппсальского университета, Швеция (Svante Janson of Uppsala University, Sweden)
(а также независимо от него, Гартом Пэйном (Garth Payne)). Я не знаю, которое из двух решений, если вообще одно из них (не третье), предполагалось изначально.


   В решении Сванте, где P -- это содержимое конкретного ведра, показывается, как можно  каждый раз уменьшать P и довести его до нуля. В моём доказательстве, напротив, показывается, как  можно  каждый раз увеличивать P до тех пор, пока одно из оставшихся вёдер не окажется пустым.


   Чтобы доказать последнее утверждение, во-первых, отметим, что мы предполагаем, что только одно ведро содержит нечётное число унций жидкости. Это верно, потому что если нет “нечётных” вёдер, то можно изменить шкалу, разделив на степень двойки. Если имеется больше двух “нечётных” вёдер, первый же шаг с двумя из них сократит их количество до одного или нуля. 


   Во-вторых, заметим, что с нечётным и чётным ведром всегда можно сделать обратный шаг, т.е. вылить половину содержимого чётного ведра в нечётное. Это потому, что каждое состояние может быть достигнуто максимум из одного состояния, таким образом, после достаточного количества шагов вы по циклу должны вернуться в начальное состояние.
Состояние прямо перед тем, как вы возвращаетесь к начальному, является результатом вашего “обратного шага”.


  И последнее, мы утверждаем, что, пока нет пустого ведра, содержимое нечётного ведра всегда можно увеличить.Если есть ведро, у которого число унций содержимого делится на 4, то можно половину его перелить в нечётное ведро. Если такового нет, его можно получить, совершив одну операцию между двумя чётными вёдрами.


   Вот доказательство Сванте, в его собственном изложении:
  “Обозначим количество унций жидкости, которое изначально содержалось в вёдрах A, B и C, --  a, b и c, где . Я опишу последовательность шагов, приводящих к такому состоянию, когда минимальное из трёх количеств жидкости меньше, чем a. Если минимум равен нулю, задача решена, в противном случае, мы переобозначаем вёдра и повторяем шаги.
  Пусть b = qa + r,  где    и  -- целое число. Запишем  в бинарном виде:
, где каждое qi это 0 или 1 и qn = 1.
   Сделаем n + 1 шагов, пронумеровав их  0, …. , n следующим образом: на шаге i мы выливаем из B в A, если qi = 1, и из C в A, если  qi = 0. Поскольку мы всё время льём в A, его содержимое каждый раз удваивается, так что А содержит  перед  i - м шагом.
Oтсюда общее количество жидкости, вылитое из B равно qa, таким образом, в конце в B остаётся .  В итоге, заметим, что общее количество жидкости, вылитое из C равно,  максимум
                    ,


Таким образом, в C  (и в B) всегда будет достаточно жидкости, чтобы сделать эти шаги.”


  Насколько мне известно, никто не знает даже приблизительно, сколько требуется шагов для решения этой задачи (при наихудшем начальном состоянии, учитывающим общее количество n унций жидкости). Моё решение показывает, что достаточно порядка n^2 шагов, тогда как  Сванте достигает лучшего результата, ограничивая число шагов произведением константы и nlogn. Точный ответ может быть ещё меньше.




ФИШКИ ПО УГЛАМ (PEGS ON THE CORNERS)


    На эту симпатичную задачу обратил моё внимание Миккель Торуп из лабораторий AT&T (Mikkel Thorup of AT&T Labs), который услышал её от Ассафа Наора (Assaf Naor)(в то время научного сотрудника Майкрософта), который услышал её от аспирантов  Еврейского университета в Иерусалиме.


    Заметим, что если фишки начинают ходить с вершин решётки (т.е точек плоскости с целыми координатами), то они всегда будут оставаться в вершинах решётки. 
   В частности, если изначально они располагаются в вершинах единичного квадрата решётки, то они, разумеется, не могут позже оказаться в углах меньшего квадрата, поскольку в решётке не существует квадратов, меньше единичного. Но почему не в углах квадрата большего?
  Основное наблюдение: прыжок через фишку -- обратимый ход! Если вы можете прийти к большому квадрату, вы можете обратить процесс и завершить ход на меньшем квадрате, что, как мы знаем, невозможно.


ФИШКИ НА ПОЛУПЛОСКОСТИ (PEGS ON THE HALF-PLANE)


  Это вариант задачи, описанной во втором томе  «Выигрышных стратегий ваших математических игр». Мы полагаем, задача была изначально придумана одним из авторов книги Джоном Конвеем.


[Winning Ways for your Mathematical Plays (Academic Press, 1982) by Elwin R. Berlekamp, John H. Conway and Richard K. Guy]    


В его варианте не разрешались прыжки по диагонали, тем не менее можно было без особых трудностей продвинуть фишку до линии y = 4, но рассуждение, подобное приведенному ниже, показывает, что позиции выше достичь невозможно.


  С прыжками по диагонали или без, трудность состоит в том, что когда фишки  поднимаются выше, вершины решётки под ними оголяются. Нам нужен такой параметр P, который бы получал награду за ушедшую высоко фишку, но для компенсации подвергался бы наказанию за оставленные позади дырки. Естественным выбором была бы сумма по всем фишкам некоторой функции от их позиции. Поскольку фишек бесконечно много, необходимо позаботиться о том, чтобы сумма сходилась.
  Можно, например, фишке на (0, y) присвоить величину r^y, где r -- некое вещественное число больше 1, и такое, что эти величины для фишек ниже оси Y в сумме дают конечное число r^y = r / (r-1). Значения в прилегающих столбцах, тем не менее, необходимо будет уменьшать, чтобы сумма по всей плоскости оставалась конечной. Если на каждом шаге при удалении от оси Y мы сокращаем на степень r, то мы получаем, что вес фишки в точке (x, y)  равен r^(y - ), и тогда общий вес в начальной позиции 


 r/(r-1) + 1/(r-1) + 1/(r-1) + 1/r(r-1) +  1/r(r-1) + … +(r^2+r)/(r-1)^2 < .


   Если фишка прыгнула, тогда при наилучшем варианте (когда прыжок был совершён по диагонали вверх к оси Y), P приобретает vr^4 и теряет v+vr^2, где v - вес фишки перед прыжком. До тех пор, пока r не больше, чем квадратный корень “золотого сечения” , что удовлетворяет , этот прирост не может быть положительным.
   Если мы продолжим и присвоим r = , тогда начальное значение P составит примерно 39.0576, но вес фишки в точке (0, 16) сам по себе равен  Поскольку мы не можем увеличить P, то следовательно, мы не можем продвинуть фишку на точку (0, 16).
Но если бы мы смогли продвинуть фишку на любую точку на линии  y = 16 или выше, тогда  мы бы смогли попасть и в точку (0, 16), остановив фишку, дошедшую до точки (x, 16), и, применяя заново весь алгоритм, сдвинуть её слева направо шагом.


   Мы не знаем наибольшего значения y точки (0, y), до которой может дойти фишка при разрешённых диагональных прыжках. Вполне возможно, какой-нибудь трудолюбивый читатель сможет устранить этот пробел.


ФИШКИ НА КВАДРАТЕ (PEGS IN A SQUARE)


   Существует несколько способов решения данной задачи, которая является частью задачи, представленной на Международной Математической Олимпиаде 1993 года.
Приведённое ниже доказательство мне рассказал Бенни Судаков из При́нстонского университета.
  Покрасим вершины (x, y) решётки  в красный цвет, если ни x,  ни y не делятся на 3, в противном случае -- в белый цвет. Получается периодический узор квадратов 2 x 2 (см. рисунок).
   Если две соседние (ортогонально) фишки стоят обе в красных вершинах решётки или обе в белых, то фишка, оставшаяся после прыжка, окажется в белой вершине. Если же одна вершина красная, а другая белая, то, напротив, фишка, оставшаяся после прыжка, будет стоять в красной вершине. Из этого следует, что если в красных вершинах находится чётное число фишек, то, при любом начальном положении, это свойство будет сохраняться, независимо от совершённых прыжков.


                          Рис. Стр 87


  Несложно увидеть, что если взять квадрат с фишками 3 x 3 и поместить его на плоскость, то в какое бы место решётки мы его не определили, он всегда накроет чётное количество красных вершин. А так как квадрат n x n, где  n кратно 3, состоит из подобных квадратов, то в нём также всегда будет содержаться чётное количество красных вершин.
Предположим, что, тем не менее, было бы возможно уменьшить количество фишек в таком квадрате до одной. Тогда мы смогли бы передвинуть начальный квадрат так, чтобы выжившая фишка оказалась в красной вершине, и это противоречие завершает доказательство.


  Существует стандартный, хотя не особенно простой или красивый(enlightening), способ доказательства того, что если n не делится на 3, то возможно уменьшить количество фишек в квадрате n x n до одной. На олимпиаде участников просили точно определить, при каких n квадраты можно  свести к одной фишке  -- довольно трудно сделать это вот так сразу!




КУЛЬБИТЫ МНОГОУГОЛЬНИКА  (FLIPPING THE POLYGON)


   Данная задача является обобщением задачи, появлявшейся на Международной Математической Олимпиаде 1986 года (представленной, как мне говорили, составителем из восточной Германии) и впоследствии получившей название “ Задача о пентагоне”.
  У задачи существует много решений, и более того, её можно обобщить и дальше, от n-угольников до произвольных связных графов. Однако, решение, приводимое ниже, выделяется среди прочих сочетанием  элегантности и строгости доказательства. Его придумали, независимо друг от друга, хотя бы два математика, один из них -- Бернар Шазель, профессор информатики Принстонского университета. (Bernard Chazelle, Professor of Computer Science at Princeton University)


   Пусть x(0), … ,x(n-1) -- числа при вершинах, дающие в сумме s > 0, с индексами, взятыми по модулю n.  Определим двустороннюю бесконечную последовательность
b(.),  где b(0) = 0 и b(i) = b(i -1) + x(i mod n). Последовательность b(.)  не является периодической, но она периодически возрастает: b(i + n) = b(i) + s.
  Если x(i) -- отрицательно, b(i) < b(i -1), и, меняя знак у x(i), мы получаем тот же  эффект, как при замене b(i) на b(i - 1), так что они располагаются теперь в возрастающем порядке. Это верно и для всех пар b(j), b(j - 1), сдвинутых на числа, кратные n. Таким образом, изменение знака у вершин сводится к сортировке (sorting) b(.), используя соседние перестановки !
  Чтобы проследить за ходом процесса сортировки, нам нужен некий конечный параметр P, который  измеряет степень, при которой b(.) нарушает порядок (out of order). Для нахождения его, положим i^+ -- число индексов j > i, для которых b(j) < b(i), и i^- -- число индексов j < i, для которых b(j) > b(i).  Обратите внимание, что i^+ и  i^-  имеют конечное значение и зависят только от i mod n. Также отметим, что  И пусть эта сумма  будет нашим волшебным параметром P.
  Когда x(i +1) меняет знак, i^+ уменьшается на 1, а все другие j^+ не изменяются. Когда P достигaeт 0, последовательность полностью отсортирована, так что все числа у вершин неотрицательны, и процесс прекращается.


                                  Рис. Стр 89.


  Мы доказали больше, чем спрашивалось: независимо от выбора чисел, процесс прекращается за одно и тоже (P)  количество  шагов, более того, конечная конфигурация также не зависит от выбора! Причина этого заключается в том, что существует только один способ сортировки b(.). Когда сортировка завершена, член b(i) исходной последовательности должен оказаться в позиции i + i^+ - i^-.




ЛАМПОЧКИ ПО КРУГУ (LIGHT BULBS IN A CIRCLE)


  Данная задача является частью задачи, представленной на на Международной Математической Олимпиаде 1993 года. При неуказанном значении n лучшим способом решения будет показать (как мы это уже делали в одном из доказательств “Пустого ведра”), что само пространство состояний является циклическим.


  Во-первых, отметим, что нет опасности в том, чтобы выключить все лампочки. Если это изменение сделано в момент времени t, лампочка t всё ещё включена. Более того, если мы посмотрим на наш круг сразу после момента t, мы можем заключить, в каком состоянии были лампочки до  момента t (изменив состояние лампочки t + 1, если лампочка t включена). Поскольку число возможных состояний в круге (учитывая, какая лампочка рассматривается, a также  какие лампочки включены) конечно, то мы со временем должны будем повторить некое состояние в первый раз. Скажем, в момент времени t1 повторилось состояние, бывшее в момент t0, где t1 и t0 отличаются на число, кратное n. Но тогда в момент t1 - 1 мы уже были в том же состоянии, как и в момент t0 - 1, что является противоречием, если только момента t0 - 1 не существовало. А это значит, что t0 равно 0, и повторилось состояние, когда все лампочки включены.


ЖУКИ НА МНОГОГРАННИКЕ (BUGS ON A POLYHEDRON)


   Данная задача была представлена в статье  Антона Клячко в 1993 году.
 (“А Funny Property of Sphere and Equations over Groups”, Communications in Algebra, Vol. 21, No. 7 (1993), pp 2555-2575) 
Для решения её мы, по сути, должны сделать противоположное тому, что делали в предыдущей задаче, то есть показать, что некий параметр будет всегда меняться в одном направлении, и, таким образом, мы не сможем вернуться в исходное состояние.
  
    Для начала заметим, что вполне возможно предположить, что в начальный момент на вершинах нет ни одного жука (можно жуков слегка подтолкнуть или придержать). Можно также предположить, что жуки двигаются по одному, каждый раз проходя через вершину.
   В любой момент времени можно нарисовать воображаемую стрелку, идущую от центра каждой грани F  к  жуку с этой грани и дальше, к центру грани с другой стороны от жука.
Начав с любой грани и следуя далее по таким стрелкам, мы должны будем, в конце концов, оказаться на некой грани второй раз, завершив цикл стрелок на многограннике.
  Этот цикл  разделяет поверхность многогранника на две части. Назовём внутренней частью цикла ту часть, которую мы обходим по часовой стрелке. Обозначим P число вершин многогранника внутри цикла.
  Изначально P могло принимать любое значение от 0 до количества всех (скажем, n) вершин многогранника. Экстремальные значения получаются, когда два жука ползут по одному ребру, и, следовательно, длина цикла равна 2. В случае, когда P=0, два жука ползут по ребру навстречу друг другу, и столкновение неизбежно.
  Когда жук в цикле переходит на следующее ребро, стрелка, проходящая через него, поворачивается вправо. Вершина, через которую он переползает, бывшая до того внутри цикла, оказывается теперь снаружи. Другие вершины также могли перейти из внутренней части цикла во внешнюю, но не существует способа перейти из внешней части во внутреннюю. Чтобы это увидеть, заметьте, что новая стрелка теперь направлена внутрь цикла. У последовательности стрелок, исходящих из её конца, нет никакой возможности избежать зацикливания, они неизбежно придут к началу какой-то стрелки цикла, создав новый цикл с меньшей внутренней частью. В частности, P теперь уменьшилось хотя бы на 1. 
     
                                  Рис. Стр 91
 
     Поскольку мы никогда не сможем вернуть P  его начальное значение, нам остаётся только надеяться, что у жуков имеется страховка от несчастного случая.




ЖУКИ НА ЧИСЛОВОМ ЛУЧЕ (BUGS ON THE LINE)


    Для начала нам  нужно убедиться, что жук либо свалится с луча слева, либо уйдёт на бесконечность вправо, он не может бродить туда-сюда вечно. Для этого ему бы пришлось проходить через какие-то числа бесконечно часто. Пусть n -- наименьшее из таких чисел.
Здесь необходимо отметить, что попадая каждый третий раз на число n, жук увидит там красный свет, и следовательно, будет вынужден пойти влево на n-1, а это противоречит предположению, что он посетил n-1 только конечное число раз.
   Теперь, когда с этим всё ясно, будет полезным думать о зелёном свете, как о 0, о красном, как о 1, и о жёлтом, как это ни парадоксально, как о “цифре” ½. A расположение цветов лампочек может быть представлено как число между 0 и 1, записанное в двоичной системе
                        x = .x1x2x3… ,


 то есть 
                 . 
                   


Будем думать о жуке на числе i, как о дополнительной “1”  к i-й позиции, определяемой 
 
                              .


Смысл этого упражнения в том, что y является инвариантом, то есть он не меняется, когда жук передвигается. При уходе жука вправо от точки i , значение числа, на котором он сидел,  поднимается на ½. Следовательно, x увеличивается на , а собственное значение жука уменьшается на ту же самую величину. Если жук идёт налево от i, он увеличивает своё значение на , но тогда  x  уменьшается для компенсации на целую цифру в i-й позиции.
   Исключение только, когда жук сваливается с луча слева, и в этом случае и x, и собственное значение жука  уменьшаются на ½, то есть  общая потеря равна 1. Когда выставляется следующий жук, y увеличивается на ½. Другими словами, значение x поднимается на ½, если выставляется новый жук, и он исчезает справа на бесконечности; и x падает на ½, если выставляется новый жук, и он сваливается с луча слева.
   Безусловно, x всегда должен лежать в единичном интервале. Если его начальное значение лежит строго между 0 и ½, жуки должны будут уходить направо, налево, направо, налево. Если x лежит между ½ и 1, жуки пойдут налево, направо, налево, направо.
   Оставшиеся случаи можно проверить руками. Если изначально x = 1 (все точки -- красные), первый жук переключит 1 на зелёный и свалится слева. Второй жук, вихляя, удалится вправо на бесконечность, оставив все лампочки опять красными, то есть чередование будет налево, направо, налево, направо. Если x = 0 изначально (все точки -- зелёные), жуки начнут уходить направо, опять направо (так как точки  меняются на все жёлтые, а потом на все красные), и затем налево, направо, налево, направо, как и прежде.
   -- самый интересный случай, потому что существует несколько способов представления 1/2 в нашей модифицированной двоичной системе: x может быть весь только ½ -е, или он может начинаться с любого конечного числа (включая 0) ½-х, за которым следуют либо 0111…, либо 1000… . В первом случае начинающий жук переключит все жёлтые лампочки на красные по мере удаления направо, таким образом мы получаем чередование  направо, налево, направо, налево. Второй случай такой же, первый жук, вихляя, уходит направо, опять оставляя все точки после себя красными. 
В третьем случае жук переключает жёлтый на красный по ходу движения, но когда он достигает красной точки, он разворачивается и двигается налево, меняя красный на зелёный по пути, пока не свалится с луча на левом конце. После этого мы приходим к случаю  x = 0, так что конечное чередование будет налево, направо, направо, налево, направо, налево, направо.
   Возвращаясь опять ко всем случаям, мы видим, что, действительно, всякий раз, когда второй жук уходит налево, третий жук уходит направо.


 Анде Холройд (Университет Британской Колумбии) и Джим Пропп (Висконсинский университет) проделали этот элегантный анализ на встрече группы “Институт Элементарных Исследований” в Баннфе, Альберта в 2003 году. 


[ https://www.birs.ca/files/scientific-reports/BIRS_SR_2003.pdf  стр.63 ]


Пропп предложил использовать жука, чтобы детерминизированно смоделировать случайные блуждания на неотрицательных целых числах, в которых шаги делаются (независимо) влево с вероятностью ⅓  и вправо с вероятностью  ⅔. В подобных блужданиях  конкретный жук падает с луча слева или убегает на бесконечность вправо с одинаковой вероятностью. Но как мы видели, детерминизированная модель показывает вместо этого, что  после первой пары жуков направления строго чередуются.
Доказательство может быть обобщено и на другие виды случайных блужданий.




КАК РАЗЛОМАТЬ ШОКОЛАДКУ (BREAKING A CHOCOLATE BAR)


 Эта до смешного простая задача известна тем, что  ставила в тупик некоторых очень крутых математиков,  которые целый день стонали и бились  головой о стены, пока к ним не приходило озарение.
  Эта до смешного простая задача известна тем, что некоторые очень крутые математики зависали на ней аж на целый день, пока среди  стенаний и битья головой об стену на них не снисходило озарение. Рискуя прослыть садистом, я пропускаю  доказательство.


Эта до смешного простая задача известна тем, что некоторые очень крутые математики зависали на ней аж на целый день, стенали и бились головой об стену, пока на них не снисходило озарение. Рискуя прослыть садистом, я пропускаю  доказательство.