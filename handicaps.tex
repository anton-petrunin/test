\chapter*{Алгоритмы с препятствиями}
\addcontentsline{toc}{chapter}{Алгоритмы с препятствиями}

\setlength{\epigraphwidth}{.85\textwidth}
\epigraph{У кислоты есть три побочных эффекта: усиление долговременной памяти, ослабление кратковременной, а третий я забыл.}{---Тимоти Лири (1920--1996)}

Старый флоридский анекдот:
два старичка, Сэм и Тед, сидят на крыльце у Сэма и болтают.

--- Это ужасно, --- говорит Тед, --- последнее время у меня совсем плохо с короткой памятью.
Я с трудом вспоминаю, принял ли я уже таблетки за этот день или нет.

--- Знаю, о чём ты говоришь, --- отвечает Сэм.
--- Мой доктор нашёл решение этой проблемы --- он добавил специальную таблетку памяти к моим ежедневным лекарствам, и теперь у меня всё отлично!

--- Ты шутишь! Как называются эти таблетки? Может, и мне такие пропишут?

--- Хм-м, хороший вопрос.
Дай подумать, ... ум-м ... быстро, скажи название растения.

--- Растения? Ты имеешь ввиду куст или дерево?

--- Нет, поменьше такое, декоративное...

--- Цветок?

--- Да, бывает красный ...

--- Гвоздика? Тюльпан?

--- Нет, такой, с шипами ...

--- Роза?

--- Точно! Она! --- Сэм поворачивается и кричит в открытую дверь, --- Роза! Как называются мои таблетки для памяти?

\medskip

В этой главе мы рассматриваем задачи на алгоритмы с произвольно странными ограничениями, обычно связанными с памятью.
Требуется достаточно много воображения, чтобы разобраться с такими задачами, и найти решение, которое сможет применить менее способное существо.
Вводную задачу в этой компании можно охарактеризовать как вполне реалистичную.

\subsection*{Нахождение числа}% (FINDING THE MISSING NUMBER)
\rindex{Нахождение числа}

Вам читают все числа от 1 до 100, кроме одного.
Каждые 10 секунд называется одно число, в произвольном порядке.
Вы хорошо соображаете, но обладаете обычной памятью, и нет возможности делать записи во время чтения.
После прочтения вам надо определить неназванное число.
Как это сделать?

\paragraph{Решение:} Легко --- вы складываете называемые числа, прибавляя их по порядку к общей сумме.
Сумма \emph{всех} чисел от 1 до 100 --- это 100, помноженное на их среднее ($50\tfrac12$), а именно 5050.
Вычитая отсюда полученную вами сумму, получаем пропущенное число.

И нет необходимости держать в уме сотни или тысячи во время суммирования, достаточно складывать по модулю 100.
В конце надо будет вычесть результат из 50 или 150, чтобы получить ответ в правильном диапазоне.
\heart

Обработка потока информации при ограничениях на возможности вычислений и память даёт множество задач.
Первая задача напоминает вводную, но она возникла из серьёзной задачи в теории вычислений.

\subsection*{Определение большинства}% (IDENTIFYING THE MAJORITY)
\rindex{Определение большинства}

Читается длинный список имён, некоторые имена повторяются много раз.
Ваша цель --- назвать имя, которое называлось большее всех остальных,
конечно если таковое существует.

\medskip

При этом, у вас в распоряжении только один счётчик, плюс, можно держать в уме только одно имя за раз.
Возможно ли это сделать?

\medskip

Следующая задача пришла ко мне от Джона Конвея, профессора Принстонского университета (наряду со \emph{многими} другими его достижениями, он является автором игры «Жизнь»).
Говорят, что одна несчастная жертва, которой попалась эта головоломка, просидела без движения на стуле шесть часов.

\subsection*{Обездвиживатель Конвея}% (THE CONWAY IMMOBILIZER)
\rindex{Обездвиживатель Конвея}

Три карты, туз, король и дама, выкладываются в рубашкой вниз на стол в несколько или в одну из отмеченных колод («левая», «средняя», «правая»).
Если они все кладутся в одну колоду, то видна только верхняя карта;
если в две колоды, то видны две карты, и вы не знаете, под которой из двух спрятана третья.

Ваша задача --- сложить все карты в левую колоду ---  сверху туз, потом король, а внизу дама.
Можно перекладывать по одной карте за ход, снимая только верхнюю карту с одной из колод и кладя её на другую (возможно пустую).

Проблема в том, что у вас нет кратковременной памяти, и вам необходимо придумать некий алгоритм, в котором каждый ход основывается только на том, что видно в данный момент, а не на том, что вы видели и делали перед этим, или сколько произошло ходов.
Сторонний наблюдатель сообщит вам, когда вы выиграете.
Возможно ли придумать алгоритм, приводящий к выигрышу за ограниченное число шагов, вне зависимости от начальной раскладки карт?

\medskip

Две из оставшихся четырёх головоломок о выключателях --- очень полезных приспособлениях для придумывания головоломок.
Последняя полушутошная задача напомнит анекдот в начале главы.

\subsection*{Крутящиеся выключатели}% (SPINNING SWITCHERS)
\rindex{Крутящиеся выключатели}

К лампочке последовательно подключены четыре одинаковых, ничем не помеченных выключателей.
Эти выключатели --- простые кнопки, по виду которых невозможно определить, в каком они находятся состоянии.
Их состояние можно поменять, нажав на кнопку.
Выключатели установлены в углах вращающегося квадрата.
За один раз вы можете одновременно нажать любое количество кнопок, а затем ваш противник крутит квадрат.
Докажите, что существует детерминистический 
алгоритм, который позволит вам включить лампочку за не более чем фиксированное число шагов.

\subsection*{Комната с одной лампочкой}% (THE ONE-BULB ROOM)
\rindex{Комната с одной лампочкой}

Каждого из $n$ заключённых посылают в некую комнату бесконечно часто, но в порядке, определяемом тюремщиком.
У заключённых есть возможность договориться заранее, но как только начинаются посещения комнаты, у них остаётся единственный способ общения --- включать или выключать лампочку в комнате.
Помогите создать протокол, который в итоге позволит \emph{кому-то} из заключённых определить, что каждый из них побывал в комнате.

\subsection*{Два шерифа}% (ТHE TWO SHERIFFS)
\rindex{Два шерифа}

Шерифы двух соседних городков ищут убийцу, в деле имеется восемь подозреваемых.
Опираясь на независимые достоверные оперативные источники, каждый шериф сузил свой список подозреваемых до двух человек.
И теперь они ведут телефонный разговор, его цель --- сравнить информацию, и если их пары подозреваемых имеют одно пересечение, то арестовать убийцу.

Проблема в том, что их телефон прослушивается местной бандой линчевателей, которым известен начальный список подозреваемых, но к каким парам сошлись шерифы, они не знают.
Если в результате этого звонка линчеватели смогут точно определить, убийцу, то его линчуют прежде, чем смогут арестовать.

Могут ли шерифы, которые раньше никогда не встречались, вести разговор таким способом, чтобы по его окончанию они оба знали, кто убийца (если это возможно), а при этом банда линчевателей осталась бы в неведении?

\subsection*{Рассеянный профессор}% (THE ABSENT-MINDED PILL TAKER)
\rindex{Рассеянный профессор}

Рассеянный профессор математики должен каждый день принимать таблетку, но у него проблема с кратковременной памятью; он не помнит, принимал ли он в этот день таблетку или нет.
Чтобы помочь себе, он купил специальную коробочку с семью прозрачными ячейками, помеченными \textsc{вс}, \textsc{пн}, \textsc{вт}, \textsc{ср}, \textsc{чт}, \textsc{пт}, \textsc{сб}.
Поскольку ему приходится читать лекции, он всегда знает, какой сегодня день недели.

Проблема в том, что он получает новую упаковку с 30 или около того таблетками, как только он заканчивает старые, и это может случиться в любой день недели.
Он хотел бы сразу сложить все таблетки в коробочку, и при этом не запоминать, сколько таблеток было в упаковке, или в какой день недели он получил новые таблетки.

Кажущийся очевидным способ раскладывать таблетки по одной за раз, начиная с текущего дня, не работал, потому что наступал момент, когда в каждой ячейке было одинаковое число таблеток, и профессор не мог понять, принимал ли он в этот день таблетку или нет.
Профессор пытался класть \emph{все} таблетки в одну ячейку текущего дня, а затем перекладывать их все вправо каждый раз, когда пил таблетку.
Но при этом он иногда забывал их перекладывать!

Можно ли предоставить алгоритм, который, опираясь только на день недели и то, что видно в коробочке,  будет говорить должен ли профессор принять таблетку или нет, и если да, то из какой ячейки?
Алгоритм должен сказать профессору, как раскладывать новые таблетки
так, чтобы после этого их не нужно было перекладывать.
