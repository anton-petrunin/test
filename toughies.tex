\chapter*{Крепкие орешки}
\addcontentsline{toc}{chapter}{Крепкие орешки}

\setlength{\epigraphwidth}{.6\textwidth}
\epigraph{Задай  труднейший  из  вопросов!  и  смотри...\\
Ответ  прекрасный  возродится  изнутри!}{---Мавлана Джалал ад-Дин Мухаммад Руми «Радость   при   неожиданном разочаровании»
}

Задачи этого раздела сложны, но ст\'{о}ят потраченных усилий.
Некоторые из них являются вариациями или продолжениями уже рассмотренных задач.

\medskip

Следующая задача была сформулирована
Китом Кёрнесом %K. A. Kearnes
и Эмилем Киссом.%Emil Kiss
\footnote{K. Kearnes, E. Kiss, ``Finite algebras of finite complexity''. \emph{Discrete Math.} 207 (1999), no. 1-3, 89--135.}
Петар Маркович привёз эту задачу на конференцию в честь Даниила Клейтмана, проходившей в Массачусетском технологическом институте в 1999 году.
Нога Алан, Том Боуман, Рон Хольцман и сам Даниил Клейтман решили её на той же конференции.

Конечно же, и эту задачу ст\'{о}ит попробовать решить самостоятельно, но нет причин расстраиваться, если у вас не получится.

\subsection*{Ящики с подъящиками}
\rindex{Ящики с подъящиками}

Зафиксируем положительное целое число $n$.
\emph{Ящиком} будем называть декартово произведение $n$ множеств;
если даны $n$ множеств $A_1,\dots,A_n$, то их ящик $A_1{\times}\dots{\times}A_n$ есть множество всех последовательностей $a_1,\dots,a_n$ таких,
что $a_i$ лежит в $A_i$ для каждого~$i$.

Ящик $\bm{B}=B_1{\times}\dots{\times}B_n$ называется \emph{собственным подъящиком} $\bm{A}\z=A_1{\times}\z\dots{\times}A_n$, если $B_i$ является собственным подмножеством $A_i$ для каждого~$i$.

Возможно ли разбить какой-нибудь ящик на менее чем $2^n$ собственных подъящиков?

\paragraph{Решение:}
Найти разбиение на $2^n$ собственных подъящиков легко; следует конечно предположить, что каждое множество $A_i$ имеет хотя бы два элемента.
Однако ни один участник конференции не смог предъявить разбиение на менее чем $2^n$ подъящиков; ниже будет показано, что это сделать невозможно.

Рассмотрим один из сомножителей $A_i$. 
Выберем в нём собственное подмножество $B_i$.
Выберем в $A_i$ случайное подмножество $C_i$ с нечётным числом элементов ($C_i$ может совпасть с $A_i$ если $|A_i|$ нечётно).

Заметим, что вероятность того, что $|C_i\cap B_i|$ нечётно, равна~$\tfrac12$.
Действительно, 
$C_i$ можно получить, выбирая элементы $A_i$ по порядку так, чтобы предпоследний элемент лежал в $B_i$, а последний в его дополнении $A_i\backslash B_i$.
Каждый раз решение можно принимать, подбрасывая монетку, и только последний элемент следует выбрать так, чтобы число $|C_i|$ получилось нечётным.
В этом случае последний бросок монеты определяет чётность $|C_i\cap B_i|$.

Заметим, что ящик $\bm{C}=C_1{\times}\dots{\times}C_n$ имеет нечётный размер тогда и только тогда, когда $|C_i|$ нечётно для каждого~$i$.
Пусть $\bm{C}=C_1{\times}\dots{\times}C_n$ есть случайный подъящик нечётного размера в $\bm{A}$, то есть $C_i\subset A_i$ для каждого~$i$.
В этом случае вероятность того, что данный подъящик $\bm{B}=B_1{\times}\dots{\times}B_n$ пересекается с $\bm{C}$ по нечётному числу элементов равна $\tfrac1{2^n}$.

Предположим теперь, что существует разбиение на менее чем $2^n$ подъящиков $\bm{B}(1),\dots,\bm{B}(m)$.
Заметим, что вероятность того, что $\bm{C}$ пересекается с каждым подъящиком $\bm{B}(i)$ по чётному числу элементов положительна (она не меньше чем $1-\tfrac{m}{2^n}$),
но это невозможно, поскольку $\bm{C}$ содержит нечётное число элементов.
\heart

Для тех отважных сердец, что всё ещё с нами, мы предлагаем больше таких задач.
Начнём с задачи Сары Робинсон, которая попала в «Нью-Йорк таймс».\footnote{S. Robinson ``Why Mathematicians Now Care about their Hat Color''. \emph{The New York Times}, April 10, 2001.}

\subsection*{Отгадать цвет шляп}
\rindex{Отгадать цвет шляп}

Команда шляпников вернулась.

На этот раз цвет шляпы каждого игрока определяется подбрасыванием монеты.
Игроки становятся в круг так, чтобы видеть цвета шляп остальных, никакого общения не допускается.
Далее игроков отводят в сторону и предоставляют шанс отгадать цвет своей шляпы --- он может быть синим или красным;
но им также предоставляется право \emph{молчать}.

Развязка ужасна: если все молчали, или хотя бы один назвал неверный цвет, то всех игроков казнят.
Может показаться, что лучшей стратегией будет молчать всем, кроме одного, в этом случае шансы выжить будут $50\%$.
Но, поразительным образом, 15 игроков могут добиться выживания в более чем $90\%$ случаев. 
Как это сделать?

Если вам кажется, что улучшить шансы в $50\%$ невозможно, то скорее всего вы правильно поняли условие задачи.
Но прежде чем отчаиваться, попробуйте случай трёх игроков.

\medskip 

Решение следующей задачи неожиданно связано с решением предыдущей.
Далее подсказок не ждите.

\subsection*{Пятнадцать битов и шпионка}
\rindex{Пятнадцать битов и шпионка}

Каждый день шпионка имеет доступ к передаче 15-и нулей и единиц на местной радиостанции.
Это её единственный канал общения с центром.
Она не знает, как выбираются биты, но каждый день у неё есть возможность \emph{подменить} любой из них, то есть поменять его с 0 на 1 или наоборот.

Сколько информации она может передать за день?

\subsection*{Углы в пространстве}
\rindex{Углы в пространстве}

Докажите, что среди любого множества из более чем $2^n$ точек в $\mathbb{R}^n$, найдутся три, которые определяют тупой угол.

\subsection*{Два монаха на горе}
\rindex{Два монаха на горе}

Помните монаха из Главы 5 (Геометрия), который забрался на Фудзияму в понедельник, а спустился во вторник?
На этот раз он вместе с собратом-монахом поднимается на гору в один и тот же день, начиная одновременно с одной высоты, но по разным тропам.
По пути к вершине их тропы могут идти вверх и вниз, но не опускаются ниже стартовой высоты.

Требуется доказать, что монахи могут изменять свои скорости (возможно идя назад) так, чтобы в \emph{каждый} момент дня они находились бы на одной высоте!

\subsection*{Сумма под контролем}
\rindex{Сумма под контролем}

Дан список из $n$ вещественных чисел $x_1,\dots,x_n$ из отрезка $[0,1]$.
Докажите, что можно найти числа $y_1,\dots,y_n$ такие, что для любого $k$ выполняется $|y_k|=x_k$ и
\[\left|\sum_{i=1}^ky_i-\sum_{i=k+1}^ny_i\right|\le 2.\]

\subsection*{Двухламповая комната}
\rindex{Двухламповая комната}

Вы помните заключённых и комнату с одной лампой?
Теперь снова каждого из $n$ заключённых будут вызывать по одиночке в комнату, бесконечное число раз и в произвольном порядке, определяемом их тюремщиком.
Однако на этот раз в комнате есть \emph{две} лампы, каждая со своим бинарным выключателем.
Средств связи, кроме этих выключателей нет, и начальные состояния их неизвестны.
У заключённых снова есть возможность договориться заранее.

Опять же, требуется, чтобы один из заключённых в какой-то момент смог сделать вывод, что все побывали в комнате.
Вы говорите, что это \emph{уже} сделано с \emph{одним} выключателем.
Да, но на этот раз требуется, чтобы все заключённые получили одинаковые инструкции.

\subsection*{Площадь против диаметра}
\rindex{Площадь против диаметра}

Докажите, что среди всех плоских фигур единичного диаметра, круг имеет наибольшую площадь.

\subsection*{Разрез пополам}
\rindex{Разрез пополам}

Докажите, что из каждого набора из $2n$ целых чисел, можно выбрать $n$ чисел, сумма которых делится на $n$.

\subsection*{Салфетки без метрдотеля}
\rindex{Салфетки без метрдотеля}

Помните задачу про банкет, там, где куча математиков рассаживается за большой круглый стол?
И снова на столе, между каждой парой приборов, находится кофейная чашка с салфеткой.
Каждый человек, садясь, берёт салфетку слева или справа;
если обе в наличии, то он выбирает её случайным образом.

На этот раз метрдотеля нет; места занимаются в случайном порядке.
Предполагая, что число математиков велико, какая их доля (асимптотически) останется без салфеток?

\subsection*{Группа солдат в поле}
\rindex{Группа солдат в поле}

Возможно, вы также помните солдат в поле, каждый из них присматривал за ближайшим соседом.
Предположим, большое число солдат стоят в случайных позициях на большом квадрате, и они разбиты на максимально возможное число групп таких, что солдаты присматривают только внутри групп.

Чему равен средний размер группы?

\subsection*{Игреки на плоскости}
\rindex{Игреки на плоскости}

Вам уже известно, что на плоскости нет несчётного числа непересекающихся топологических восьмёрок.
Конечно же, можно найти континуум отрезков и окружностей.
Закономерный вопрос: что можно сказать про игреки (Y), то есть про подмножества топологически эквивалентные трём отрезкам с общим концом?

Докажите, что на плоскости можно нарисовать только счётное число непересекающихся игреков.

\subsection*{Снова намагниченные доллары}
\rindex{Снова намагниченные доллары}

В последней задаче мы возвращаемся к намагниченным долларам, но слегка увеличиваем их притяжение.
На этот раз бесконечная последовательность монет сыпется в две урны.
Когда одна урна содержит $x$ монет, а другая $y$, следующая монета попадёт в первую урну с вероятностью $x^{1{,}01}/(x^{1{,}01}+y^{1{,}01})$, а иначе --- во вторую.

Докажите, что с какого-то момента одна из урн не получит ни одной монеты.
