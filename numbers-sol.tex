\section*{Решения и комментарии}

\subsubsection*{Нули, единицы и двойки}%НУЛИ, ЕДИНИЦЫ И ДВОЙКИ (ZEROES, ONES AND TWOS) 

Для части а) применим знаменитый и полезный принцип ящиков Дирихле:
Если число голубей больше числа ящиков, то хотя бы в одном из ящиков находится более одного голубя.

Мы знаем, что существует только $n$ различных остатков от деления на $n$, а множество $\{1, 11, 111, 1111,\dots\}$, чей наибольший элемент
состоит из $n+1$ цифр, содержит $n+1$ членов.
Следовательно, в нём содержатся два числа, имеющих одинаковый остаток при делении на $n$.
Отнимите одно от другого!
\heart

Дэвид Гейл %(David Gale) 
указал мне на то, %обратил моё внимание на то, 
что если $n$ не делится на 2 и 5, то можно найти число кратное $n$, представляемое (в десятичной системе) одними единицами.
Действительно, приведённое выше доказательство даёт число вида $111\dots111000\dots000$; 
если у нас на конце $k$ нулей, то деление на $10^k=2^k\cdot 5^k$ оставляет нам число из одних единиц и всё ещё кратное $n$.

В части (б) легче всего, применить индукцию по $k$
и доказать, что существует число из $k$ цифр, кратное $2^k$ и состоящее только из единиц и двоек.
Добавление 1 или 2 в начало такого числа увеличит его на $2^k\cdot 5^k$ или $2^{k+1}\cdot 5^k$, в обоих случаях сохраняется делимость на $2^k$.
Так как два полученных числа отличаются на $2^k\cdot 5^k$, одно из них должно делится на $2^{k+1}$.
\heart

Первая задача попала ко мне от Муту Мутукришнана %(Muthu Muthukrishnan) 
из отдела исследований AT\&T и Ратгерского университета.
Вторая была представлена на пятой Всесоюзной математической олимпиаде в Риге в 1971 году.
Приведённое здесь решение принадлежит Саше Баргу %(Sasha Barg) 
из Мэрилендского университета.

\medskip

В похожей задаче на первой Всесоюзной математической олимпиаде в Тбилиси в 1967 году требовалось доказать, что существует число, которое делится на $5^{1000}$ и в своей десятичной записи не содержит ни одного нуля.
Один способ --- это доказательство от противного, 
пусть $k$ --- наибольшая возможная степень пятёрки.
Пусть $n$ есть кратное наибольшей степенью пятёрки (скажем $5^j$ при $j\le k$)
среди всех $k$-значных чисел.
Тогда $n\equiv c\cdot 5^j\pmod{5^{j+1}}$ для некоторого $c\in\{1,2,3,4\}$.
Поскольку $5$ является простым, существует $d\in\{1,2,3,4\}$, что $n\equiv d\cdot 10^j\pmod{5^{j+1}}$.
Отняв от $n$ число $d\cdot 10^{j}$ или прибавив к нему $(5-d)\cdot 10^{j}$, мы получаем число без нулей и улучшаем его делимость --- противоречие.
\heart

\subsubsection*{Суммы и разности}%СУММЫ И РАЗНОСТИ (SUMS AND DIFFERENCES) 

Данная задача была также представлена на пятой Всесоюзной математической олимпиаде в Риге, 1971.

\medskip

%Упорядочим числа $x_1<x_2<\dots<x_n$ (мы полагаем $n=25$, но нам %потребуется только то, что $n$ нечётно).
%Если $x_n$ не является одним из искомых чисел, значит для %каждого меньшего числа $x_i$ существует другое число $x_j$, %такое что $x_i+x_j=x_n$.
%Таким образом, первые 24 числа разбиваются на пары и %следовательно $x_i+x_{n-i}\z=x_n$ для любого $i$.

%Сумма $x_{n-1}$ с любым из чисел $x_2,\dots,x_{n-2}$ больше, чем %$x_n\z=x_1+x_{n-1}$.
%Значит, если $x_{n-1}$ не является одним из искомых чисел, то %для каждого числа $x_i$ из $x_2,\dots,x_{n-2}$ есть пара $x_j$, %что 
%$x_{i}+x_{j}=x_{n-1}$.
%Отсюда $x_{i}+x_{n-1-i}=x_{n-1}$ для каждого $i$.
%Однако в этом случае $x_{(n-1)/2}$ парно самому себе, 
%и пара $x_{n-1}$, $x_{(n-1)/2}$ является решением нашей задачи %--- противоречие.
%\heart 

Предположим множество $X$ из чисел $x_1 < x_2 < \dots < x_{25}$ даёт контрпример.
%Assume the set X, with members x_1 < x_2 < ... < x_25, is a counterexample. 
Поскольку суммы $x_{25} + x_1$, $x_{25} + x_2, ..., x_{25} + x_{24}$ не в $X$,
соответствующие разности должны быть в $X$,
и значит
\[x_1 + x_{24} = x_2 + x_{23} = x_3 + x_{22} + =\dots = x_{12} + x_{13} = x_{25}.\]
%Since the sums x_{25} + x_1, x_{25} + X_2, ..., x_{25} + x_{24} are not in X, the corresponding differences are in X;
%it follows that X_1 + x_{24} = x_2 + x_{24} = x_3 + x_{23} + ... = x_{12} + x_{13} = x_{25}.

Заметим, что суммы $x_{24} + x_2$, $x_{24} + x_3,\dots$ так же превосходят $x_{25}$,
а значит опять они не в $X$, а соответствующие разности в $X$.
%The sums x_{24} + x_2, x_{24} + x_3, ..., x_{24} + x_{23} also exceed x_{25} (since x_{25} = x_{24} + x_1), thus are not in X, so again the corresponding differences are in X.
Разница $x_{24} - x_{23}$ может равняться только $x_1$ и значит числа с $x_2$ до $x_{22}$ разбиваются на пары 
\[x_2 + x_{22} = x_3 + x_{21} = x_4 + x_{20} = ... = x_{11} + x_{13} = x_{24}.\]
%The difference x_{24} - x{23} can only be x_1, so the numbers x_2 up to x_{22} must pair up with x_2 + x_{22} = x_3 + x_{21} = x_4 + x_{20} = ... = x_{11} + x_{13} = x_{24}.
Но число $x_{12}$ остаётся без пары, что приводит к противоречию, и решению головоломки.\heart
%But that leaves out x_{12}, so no such X can exist, and the puzzle's statement is proved.

В доказательстве, число 25 можно заменить любым нечётным числом $n> 3$ (при $n=3$ множество $\{1,2,3\}$ даёт контрпример).
Если $n$ чётно, то доказательство проще. 
%The proof works for 25 replaced by any odd number n > 3 (for n=3, the set {1,2,3} is a counterexample) and a simpler version works for even n.


\subsubsection*{Генерирование рациональных чисел}% (GENERATING THE RATIONALS) 

В первую очередь заметим, что множество $S$ содержит все двоичнорациональные числа, 
то есть дроби вида $p/2^n$.
Все такие числа со знаменателем $2^n$ и нечётным числителем можно получить, взяв среднее значение двух соседних чисел со знаменателями меньшей степени.

Любая дробь $p/q$, очевидно, является средним $p$ единиц и $q-p$ нулей.
Выберем $n$ достаточно большим и заменим все нули на $1/2^n$, $-1/2^n$, $2/2^n$, $-2/2^n$, $3/2^n$ и так далее, включая один $0$, если $p$ нечётное.
Подобным же образом заменим единицы на $1-1/2^n$, $1+1/2^n$, $1-2/2^n$, $1+2/2^n$ и так далее.
Конечно, некоторые из этих чисел находятся вне единичного отрезка, но можно повторить то же рассуждение для интервала между двоичнорациональными числами, содержащего $p/q$ и лежащего строго в единичном отрезке.%(rescale the procedure to fit some dyadic interval)
\heart

Источник: Тринадцатая Всесоюзная математическая олимпиада, Тбилиси, 1979.

\subsubsection*{Суммирование дробей}%СУММИРОВАНИЕ ДРОБЕЙ ( SUMMING FRACTIONS ) 

Проведём доказательство по индукции, заметив, что высказывание верно для $n=2$.
При переходе от $n$ к $n+1$ мы добавляем $1/pn$ для каждого $p$ с $(p,n)=1$ и теряем $1/pq$ для пар $p$ и $q$ таких, что $(p,q)=1$ и $p+q=n$.
Таким образом, каждая пара, удовлетворяющая условию задачи, означает потерю $1/pq$, и добавление $1/pn+1/qn=1/pq$, а, значит, результат не меняется.\heart

Источник: Третья Всесоюзная математическая олимпиада, Киев, 1969.

\subsubsection*{Вычитания по кругу}%ВЫЧИТАЯ ПО КРУГУ (SUBTRACTING AROUND THE CORNER) 

Один внештатный преподаватель математики старших классов (средняя школа в Фер Лон, Нью-Джерси, 1962)
%(Fair Lawn Senior High School) 
рассказывал мне,
что некоторые военнопленные Второй Мировой войны развлекались тем, что брали различные наборы из четырёх чисел и смотрели как долго они могут прокручивать эту операцию.

\medskip

Обе задачи решаются рассмотрением процесса по модулю 2.
Для $n=4$ с точностью до циклических перестановок и симметрий,
1 0 0 0 
и
1 1 1 0 становятся 
1 1 0 0, затем 
1 0 1 0, затем 
1 1 1 1 и затем 
0 0 0 0.
Поскольку здесь охватываются все случаи, мы видим, что при работе с обычными целыми числами нам понадобится не более 4 шагов, чтобы все они стали чётными, и на этом этапе, до того, как продолжить далее, можно поделить их на двойку в наибольшей общей степени.
Так как самое большое в последовательности число $M$ не может увеличиваться и уменьшается в два раза или более хотя бы один раз за каждые четыре шага, 
мы приходим к 
0 0 0 0 
максимум за $4(1+\lceil\log_2 M\rceil)$ шагов.

С другой стороны, при $n=5$ последовательность 
1 1 0 0 0
(рассматриваемая как последовательность бинарных или обычных чисел)
проходит по циклу 
1 0 1 0 0, 
1 1 1 1 0, 
1 1 0 0 0.\heart 

Несложный анализ, использующий многочлены над целыми по модулю два, показывает, что всё зависит от того является ли $n$ степенью двойки.

\subsubsection*{Прибыли и убытки}%ПРИБЫЛИ И УБЫТКИ (PROFIT AND LOSS) 

Данная задача представляет собой адаптированный вариант задачи, предложенной одним вьетнамским автором на Международной математической олимпиаде 1977 года.
Мне её рассказал Титу Андрееску, %(Titu Andreescu), 
за что я ему очень благодарен.
Решение, однако, моё собственное.

\medskip

Нам нужна, конечно, последовательность чисел максимальной длины, такая, что сумма чисел каждой подстроки длины 8 --- больше нуля, а сумма чисел каждой подстроки длины 5 --- меньше нуля.
Строка, без сомнения, должна быть конечной, более того, длиной меньше 40, иначе вы могли бы выразить сумму первых 40-ка членов и как (положительную) сумму 5 подстрок длины 8, и как сумму (отрицательную) 8 подстрок длины 5.

Давайте решим эту задачу в более общем виде.
Пусть $f(x,y)$ --- длина максимальной строки, такой, что сумма каждой $x$-подстроки положительна, а сумма каждой $y$-подстроки --- отрицательна.
Можно предположить, что $x>y$.
Если $x$ кратно $y$, тогда $f(x,y)=x-1$ (мы должны согласиться, что это утверждение верно по отношению к $x$-подстрокам за отсутствием таковых).

Что, если $y=2$ и $x$ --- нечётное?
Тогда можно построить строку длины $x$ с чередующимся элементами, скажем, $x-1$ и $-x$.
Однако такой строки длины $x+1$ не существует.
Действительно, в каждой $x$-подстроке нечётный элемент должен быть положительным 
(так как можно покрыть всю $x$-подстроку 2-подстроками за исключением произвольного нечётного элемента).
Для двух $x$-подстрок, это означает, что все элементы положительны ---
противоречие.

Рассуждение выше подсказывает, что $f(x,y)\le x+y-2$, если $x$ и $y$ взаимно просты, 
то есть не имеют общего делителя отличного от~1.
Докажем это по индукции.
Рассуждая от противного, предположим что $f(x,y)\ge x+y-1$;
то есть имеется строка длины $x+y-1$, удовлетворяющая указанным условиям.
Положим $x=ay+b$, где $0<b<y$.
Заметим, что каждая $b$-подстрока %(any consecutive b of them) 
может быть представлена как $x$-подстрока полной строки, 
без $a$ штук $y$-подстрок; 
следовательно, сумма в любой $b$-подстроке положительна.
Отсюда
$f(b,y)\ge x+y-1$,
что противоречит предположению индукции поскольку $y$ и $b$ взаимно просты и $b<y<x$.

Чтобы показать, что $f(x,y)=x+y-2$, когда $x$ и $y$ взаимно просты, построим $(x+y-2)$-строку, обладающую требуемыми свойствами.
Более того, числа нашей строки будут принимать ровно два различных значения $u$ и $v$,
а также строка будет периодической с двумя периодами $x$ и $y$.

Представим себе, что мы расставили $u$ и $v$ произвольным образом в первой $y$-подстроке.
Продолжим расстановку, заставляя строку быть периодической с периодом $y$.
Чтобы добиться того же с периодом $x$,
нам нужно обеспечить, чтобы последние $y-2$ элементов соответствовали первым.
Это условие можно записать как $y-2$ равенств на $y$ значений выбранных нами изначально.
Поскольку этих равенств недостаточно, чтобы заставить все решения быть одинаковыми, можно гарантировать, что существует по меньшей мере, одно $u$ и одно $v$.

Проделаем это для $x=8$ и $y=5$.
Пусть $c_1\dots c_5$ будут первые пять элементов строки, 
таким образом, вся строка будет выглядеть как
$c_1c_2c_3c_4c_5c_1c_2c_3c_4c_5c_1$.
Для того, чтобы строка была периодична с периодом 8, 
нужно потребовать $c_4=c_1$, $c_5=c_2$ и $c_1=c_3$.
Значит, можно считать $c_1=c_3=c_4=u$ и $c_2=c_5=v$; 
тем самым получается последовательность $uvuuvuvuuvu$.

Возвращаясь к задаче в общем виде, 
заметим, что строка с периодом $x$ автоматически обладает следующим свойством: сумма каждой $x$-подстроки одна и та же, потому как каждый раз, когда вы сдвигаете $x$-подстроку на один шаг, элемент, приобретённый на одном конце, такой же, как и элемент, потерянный на другом.
Конечно, это высказывание справедливо и для $y$-подстрок, если строка периодична с периодом $y$.

Пусть $S_x$ --- сумма $x$-подстроки, а $S_y$ --- сумма $y$-подстроки.
Заметим, что $S_x/x\ne S_y/y$.
Это объясняется тем, что, если у нас $u$ встречается в каждой
 $x$-подстроке, скажем, $p$ раз, а $v$, в каждой $y$-подстроке --- $q$ раз, 
 то равенство $S_x/x=S_y/y$
 означало бы 
\[y(pu+(x-p)v)=x(qu+(y-q)v)\]
что сводится к $yp=xq$ так как $u\ne v$.
Поскольку $x$ и $y$ взаимно простые, этого не может быть при $0<p<x$ и $0<q<y$.

Отсюда следует, что можно подобрать $u$ и $v$ так, чтобы сумма $S_x$ была положительна, а $S_y$ отрицательна.
Например, в случае, рассматриваемом выше, каждая 8-подстрока содержит пять $u$ и три $v$, 
а в свою очередь, каждая 5-подстрока имеет три $u$ и две $v$.
Если взять $u=5$ и $v=-8$, то получим $S_x=1$ и $S_y=-1$.
Итак, решением исходной задачи будет последовательность $5,-8,5,5,-8,5,-8,5,5,-8,5$.
\heart

Усердный читатель легко обобщит вышеприведённое доказательство на случай, 
когда наибольший общий делитель $x$ и $y$ отличен от~1.
Результат будет $f(x,y)=x+y-1-\text{НОД}(x,y)$.

\subsubsection*{Первое нечётное число в словаре}%ПЕРВОЕ НЕЧЁТНОЕ ЧИСЛО В СЛОВАРЕ (FIRST ODD NUMBER IN THE DICTIONARY) 

Решение данной задачи --- всего лишь вопрос внимательного и методичного рассмотрения последовательности слов, составляющих числительные.
Самое первое число будет, конечно, «восемнадцать», и первое идущее за ним слово (можем считать его как бы суффиксом) --- «миллионов».
Итак, искомое число должно начинаться с «восемнадцать миллионов», следуя подобным образом далее, получаем окончательный ответ: 18 018 089 --- «восемнадцать миллионов восемнадцать тысяч восемьдесят девять».%
\footnote{Ответ в оригинальной задаче на английском языке выглядит похоже: 8,018,018,885 --- «eight billion eighteen million eighteen thousand eight hundred eighty-five».}
\heart

Идея этой забавной задачи пришла ко мне после того, как Херб Уилф из Пенсильванского университета %(Herb Wilf, University of Pennsylvania) 
попросил меня назвать первое простое число в словаре.
Автором этого вопроса считается компьютерный гуру из Стэндфордского университета Дональд Кнут. %(Donald Knuth, Stanford University).
Рассуждая так же, как и выше, с проверкой простоты числа на компьютере
получаем, что ответ равен 18 018 881.%
\footnote{На английском, это будет 8,018,018,881 --- «eight billion eighteen million eighteen thousand eight hundred eighty-one».
%на французком 5?
%на немецком 8 808 808 803?
}

