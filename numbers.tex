%ё
\chapter*{Числа}

\setlength{\epigraphwidth}{.4\textwidth}
\epigraph{Счастья светлого полны\\ %We learned to be happy,
На балу танцуем мы.\\ % We danced’ round the hall
А причину что скрывать ---\\ %And learning to count was the key to it all.
Научились мы считать.}{Граф фон Знак, «Улица Сезам»}

%The Count, “Sesame Street” 

 

Числа --- это бесконечный источник очарования, а для некоторых --- болезнь на всю жизнь. %кого-то и страсть на всю жизнь.
Бывает, что свойства конкретного числа овладевают умом человека;
про такие числа придумано множество интригующих задач,
часто требующих делать выводы из того, что на первый взгляд кажется нехваткой данных.
%основанных на кажущейся нехватке данных.

Задачи, подобранные здесь, подразумевают % предлагают 
более общий подход. % идею, более универсальный подход.
%Идея задач, подобранных здесь, --- предложить более общий, универсальный подход.
%Дух (spirit) данной коллекции ( Настроение данной коллекции подразумевает) предлагает тем не менее стремление к большей (общности )универсальности.
Наши численно-теоретические задачи о числах в целом, а не только об отдельных, особенных числах.
В ряде случаев для их решения вам понадобится несколько больше, чем знание того, 
что любое натуральное число представляется как произведение простых единственным образом.

\medskip

Вот задача для примера:

\subsection*{Двери шкафчиков}% ДВЕРИ ШКАФЧИКОВ (LOCKER DOORS)

Шкафчики в раздевалке школьного спортивного зала пронумерованы по порядку от 1 до 100.
Ученик, пришедший первым, открывает все шкафчики.
Второй ученик проходит следом и закрывает все шкафчики с чётными номерами, третий ученик меняет положение дверей у шкафчиков с номерами, кратными 3.

Так продолжается до тех пор, пока не пройдёт сотый ученик.
Какие шкафчики будут после этого открыты?
 
\paragraph{Решение:}

Состояние $n$-го шкафчика (открыт-закрыт) изменяется, когда проходит $k$-тый ученик, если $k$ --- делитель числа $n$.
Обычно, мы имеем пару делителей $\{j,k\}$, где $j\cdot k=n$.
Таким образом, совокупный эффект от учеников $j$ и $k$ на данную дверь отсутствует. %будет равен 0.
Но, если $n$ --- совершенный квадрат, то нет другого делителя, отменяющего действия $\sqrt{n}$-го ученика.
Следовательно, в конце будут открыты только шкафчики, номера которых являются совершенными квадратами: 1, 4, 9, 16, 25, 36, 49, 64, 81 и 100.\heart
 
В начале раздела мы сделаем пару наблюдений, касающихся представления целых чисел в десятичной системе счисления, и закончим неожиданно хитрой головоломкой для ужина с друзьями. % (surprisingly subtle dinner table conundrum).

\subsection*{Нули, единицы и двойки}% НУЛИ, ЕДИНИЦЫ И ДВОЙКИ. (ZEROES, ONES AND TWOS)

Пусть $n$ --- натуральное число.
Докажите, что (а) существует число кратное $n$ (не равное нулю), чьё представление в десятичной системе содержит только нули и единицы, и
(б) существует число кратное $2^n$, состоящее только из единиц и двоек.

\subsection*{Суммы и разности} %СУММЫ И РАЗНОСТИ (SUMS AND DIFFERENCES)

Даны 25 положительных чисел.
Докажите, что можно выбрать из них два числа, так, что ни одно из оставшихся чисел не равно ни их сумме, ни их разности.

\subsection*{Генерирование рациональных чисел}%??? Получение ГЕНЕРИРОВАНИЕ РАЦИОНАЛЬНЫХ ЧИСЕЛ (GENERATING THE RATIONALS)

Множество $S$ содержит 0 и 1, а также средние значения всех конечных непустых подмножеств множества $S$.
Докажите, что $S$ содержит все рациональные числа единичного отрезка.

\subsection*{Суммирование дробей}%СУММИРОВАНИЕ ДРОБЕЙ ( SUMMING FRACTIONS )

Дано натуральное число $n>1$, 
сложите все дроби $1/pq$, где $p$ и $q$ взаимно простые числа при $p+q>n$ и $0<p<q\le n$.
Докажите, что результат суммирования всегда равен $1/2$.

\subsection*{Вычитания по кругу}
%ВЫЧИТАЯ ПО КРУГУ (SUBTRACTING AROUND THE CORNER)

Напишите последовательность из $n$ положительных чисел.
Замените каждое из чисел на модуль разности %(absolute difference) 
этого и следующего по кругу за ним числа.
Повторяйте до тех пор, пока все числа станут равны нулю.
Докажите, что для $n=5$ этот процесс может продолжаться бесконечно, 
а при $n=4$ он всегда закачивается.

\subsection*{Прибыли и убытки}%ПРИБЫЛИ И УБЫТКИ (PROFIT AND LOSS)

На совещании акционеров правление представило помесячный отчёт о прибылях (либо об убытках) сo времени проведения последнего собрания.
«Заметьте, --- сказал генеральный директор,--- за каждые идущие подряд восемь месяцев мы получали прибыль»
«Может быть и так, --- посетовал один из акционеров, --- но я также вижу, что за каждый период из последовательно идущих пяти месяцев мы несли убытки!»

Какое максимальное число месяцев могло пройти сo времени проведения последнего собрания?

\subsection*{Первое нечётное число в словаре}%ПЕРВОЕ НЕЧЁТНОЕ ЧИСЛО В СЛОВАРЕ (FIRST ODD NUMBER IN THE DICTIONARY)

Представьте, числа от 1 до $10^{10}$ записаны формальным русским языком (например, «двести одиннадцать» или «одна тысяча сорок два») и затем расставлены в алфавитном порядке (как в словаре, буква за буквой, пробелы и дефисы игнорируются).
Какое нечётное число встретится первым?
