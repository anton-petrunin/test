\chapter*{Предисловие}
\addcontentsline{toc}{chapter}{Предисловие}

\setlength{\epigraphwidth}{.84\textwidth}
\epigraph{Сомнение --- это вестибюль, через который все должны пройти прежде, чем попасть во дворец Мудрости.
Когда мы пребываем в сомнении и ищем истину своими собственными силами, мы приобретаем нечто, что останется с нами и будет служить нам снова и снова.
Но, если во избежании трудностей поиска мы воспользуемся превосходными познаниями друга, это знание не задержится у нас.
Мы его не купили, а взяли взаймы.}{---Чарльз Калеб Колтон}
                                                                                     

Эти задачи не для всех.

Чтобы оценить и решить их, необходимо, но не достаточно, быть в хороших отношениях с математикой.
Вам нужно знать, что такое точка и прямая, что такое простое число, и сколькими способами можно упорядочить пять карт на руках у игрока в покер.
Но самое важное, вы должны знать, что значит \emph{доказать}.

Вам \emph{не} потребуется профессионального знания математики.
Вы знаете, что такое группы? Прекрасно ---  они вам здесь не понадобятся.
Ваш компьютер, калькулятор и учебник по матанализу могут оставаться лежать в своих ящиках, но ваш «котелок» должен варить.

Кто вы? Любители математики.
Учёные из разных областей науки.
Блестящие ученики старших классов и студенты ВУЗов.
И да, профессиональным математикам и учителям здесь тоже найдётся над чем подумать.
Эти задачи вы (обычно) не найдёте в журнальных статьях, в списке заданий для домашней работы или в других книгах головоломок.

Так где же нашёл их я? В дружеских беседах.
Среди математиков такие задачи распространяются тем же путём как расходятся анекдоты.
В некоторых случаях мне удалось отыскать опубликованный источник, такой как Всесоюзные математические олимпиады, Международные математические олимпиады или рубрики Мартина Гарднера, но, конечно, это не всегда может быть оригинальным печатным источником;
даже если так, задачи могли ходить в устном фольклоре задолго до этого.
Часто решение моё собственное и необязательно то, которое предусматривалось автором задачи.
Несколько решений предлагаются только, когда я не смог устоять перед их красотой.

Изложение задач и их решений моё собственное, и я беру на себя полную ответственность за ошибки и неточности.
Присылайте жалобы, исправления и информацию об источнике задач на электронную почту pw@akpeters.com.
(Одно исключение: как отмечено в главе 12, я не тот человек, которому надо отправлять предполагаемые решения из «Нерешённых головоломок».)

На момент написания этой книги, я уже 28 лет являюсь профессиональным математиком
(14 лет в академии и 14 лет в коммерческой компании).
Собирать математические головоломки я начал в старших классах школы в 60-х.
В этой книжке представлена сотня или около того моих самых любимых задач.
Чтобы попасть в эту книгу, задача должна удовлетворять большинству из следующих требований:

\textit{Занимательность.}
Задача должна доставлять удовольствие.
Задания с Математических олимпиад Уильяма Лоуэлла Патнема, проводящихся ежегодно для студентов колледжей в США и Канаде, созданы для проверки способностей студентов.
Это прекрасная цель, но она не всегда попадает под категорию занимательности.
(Тем на менее, в книге представлены несколько задач с олимпиад Патнема.)

\textit{Универсальность.}
Задача должна говорить о некой общей математической истине.
Сложные логические задачи, алгебраические задачи типа:
«Через два года, Алиса будет в два раза старше, чем Боб, когда он был...», задачи, основывающиеся на свойствах особенно больших  чисел, и множество других типов искусно придуманных задач исключены.

\textit{Элегантность.}
Задача должна  просто и легко формулироваться.
Ведь, чтобы передаваться устно, нужно, чтобы она легко запоминалась! Ещё лучше, когда в формулировке присутствует элемент неожиданности.

\textit{Трудность.}
Должно быть неочевидно, как решить задачу.

\textit{Решаемость.}
Задача могла бы похвастаться хотя бы одним решением, которое является простым и понятным.

Последние два пункта порождают противоречие: задача должна иметь простое решение, однако непросто решаться.
Должно быть, как в хорошей загадке, в которой трудно отгадать ответ, но легко его понять.
Конечно, нерешённые задачи из главы 12 очень сложные и им  надо простить последнее требование.

О формате книги.
Для удобства, задачи довольно свободно разделены по типу условия или решения по главам, соответствующим той или иной области математики.
Решения приводятся в конце каждой главы (за исключением последней).
Конец решения отмечен сердечком ($\heartsuit$).
Если известны происхождение и история задачи, то они представлены там же.
Условие задачи \emph{не} повторяется перед её решением.
Я хотел бы чтобы читатели попробовали решить задачи самостоятельно, прежде, чем заглянули в ответы.

Эти головоломки трудные.
Несколько из них существовали долго как нерешённые проблемы, пока кто-то не нашёл (элегантное) решение, которое вы здесь прочтёте.
Нерешённые задачи в конце книги, представляющие как бы логическое завершение коллекции, возможно только немногим сложнее, чем все остальные.

Если вы сами решили какую-то из этих задач, то можете по праву гордиться, особенно, если ваше решение лучше моего.

Удачи!

%\begin{flushright}
%--- Петер Винклер
%\end{flushright}

%Comment to Russian readers:

%  Since many of my best puzzles originated in Russia, I feel as though this Russian-language edition is to some extent "bringing coals to Newcastle." 
%But I am confident that most readers will find here lots of great puzzles they haven't seen before, and will also from the great job the translators have done in correcting errors from the English original.
% Happy puzzling!

%    ---Peter Winkler

\subsubsection*{Замечание к русскому изданию}

Поскольку многие из моих любимых головоломок родом из России, издавать эту книжку на русском языке в некотором смысле как «ехать в Тулу со своим самоваром».
Тем не менее я уверен, что большинство читателей найдут здесь множество замечательных и новых для них задач, а также извлекут пользу из работы переводчиков над исправлением ошибок в английском оригинале.

Успешного головоломания!

\begin{flushright}
---Питер Винклер
\end{flushright}

