ДЕТЕРМИНИРОВАННЫЙ ПОКЕР (DETERMENISTIC POKER)


   Немного о порядке комбинаций карт для данной задачи, и так, наилучшей комбинацией здесь является стрит-флэш (любые пять карт одной масти по порядку).
Стрит-флэш, начинающийся с туза (его также называют “Роял-флэш”) бьёт стрит-флэш, возглавляемый королём, и так далее вниз по-порядку.
   Это означает, что если Бобу удастся собрать роял-флэш, Алисина песенка спета. Поэтому, чтобы получить шанс выиграть, Алисин первый набор карт должен содержать по одной карте каждого из четырёх возможных роял-флэшев.
   Для этой цели лучшими картами являются десятки каждой масти, поскольку они не дают собрать стрит-флэш, начинающийся с десяти или с более высокой карты. И действительно,  поразмыслив минуточку, вы поймёте, что любая комбинация Алисы с изначальными четырьмя десятками выигрывает. У Боба не остаётся надежды собрать стрит-флэш выше девятки. Но чтобы не дать Алисе собрать роял-флэш, он должен взять, по меньшей мере, по одной карте выше десятки из каждой масти, то есть только одну карту меньше десяти. Алиса же теперь может поменять четыре карты и собрать стрит-флэш, начинающийся с десятки, в любой масти, кроме масти младшей карты Боба,
и Боб оказывается тут совершенно беспомощен.


   У Алисы имеются также и другие выигрышные комбинации. Эта странная игра была опубликована в одной из ранних рубрик  Мартина Гарднера.




 ШВЕДСКАЯ ЛОТЕРЕЯ (SWEDISH LOTTERY)


  Предположим, k -- наибольшее число, которое готов выбрать каждый из игроков. Если игрок выбирает k, он выигрывает, когда совпадает выбор двух других игроков, кроме случая, когда их выбор также -- k. Но если он выбирает k+1, он выигрывает каждый раз, когда выбор двух других совпадает, точка. Отсюда следует, что k+1 -- выбор лучше, чем k, и мы не можем реализовать равновесие стратегий (нарушается равновесная ситуация). 
Это противоречие указывает на то, что необходимо рассматривать произвольно большие числа -- иногда нужно выбрать 1487564.


 Для создания равновесной ситуации каждый игрок должен выбрать число j с вероятностью где 


        r = длинная формула стр 98


что примерно равно 0,543689. Вероятности выбора 1, 2, 3 и 4  равны примерно 0,456311, 0,248091, 0,134884 и 0,073335, соответственно.


  На эту довольно симпатичную идею лотереи обратил  моё внимание Улле Хэггстрём из университе́та Ча́лмерса в Гётеборге, Швеция (Olle Häggström of Chalmers University in Göteborg, Sweden.) Я не знаю, была ли эта идея осуществлена или хотя бы серьёзно рассматривалась в качестве  официальной национальной лотереи,  а вы не считаете, что должна бы?




 БЛИНЫ (PANCAKES)


  Положим, что в настоящий момент в стопках m и n блинов, и m > n. Если соотношение размеров стопок  r = m/n лежит строго между 1 и 2, то на следующем ходу новое соотношение равно . Эти пропорции равны только для  -- золотое сечение. Поскольку  -- иррациональное число, одно из двух соотношений r  и  должно превышать , а другое быть меньше 
   Первый игрок (Алиса) выигрывает только тогда, когда начальное соотношение большей стопки к меньшей превышает  Чтобы это увидеть, предположим , но m не кратно n. Запишем  , где. Тогда либо , и в этом случае Алиса ест блинов, либо , и в этом случае она съедает только  штук.
Бобу остаётся  с соотношением, меньшим , и он вынужден сделать ход, который возвращает соотношению значение больше 
   Рано или поздно, Алиса достигнет момента, когда её соотношение  -- целое число, и тогда она сможет уравнять число блинов в двух стопках  и оставить Боба с сырым блином. Заметим, что она также может, если захочет, забрать себе всю кучу блинов.
   Разумеется, если вместо этого  Алисa имеет дело  с соотношением , которое лежит строго между 1 и , она оказывается в безвыходном положении ( дело её -- швах) и теперь Боб  задаёт игру. 
    Мы приходим к заключению, что независимо от того, в какой вариант “Блинов” играют, если высоты стопок , то Алиса точно выигрывает, когда . Только в тривиальном случае, когда стопки изначально одинаковые, цель игры имеет значение.


   Данная задача была представлена на 12-й Всесоюзной математической олимпиаде в Ташкенте в 1978 году. Мне её показал Билл Газарч  из Мэрилендского университета (Bill Gasarch of the University of Maryland).




ОПРЕДЕЛЕНИЕ РАЗНОСТИ (DETERMINING A DIFFERENCE)


   Запишем разность как , где  и . Обозначим  результат замены оставшихся звёздочек в на нули в любой момент в игре, аналогично  для  и  . Алиса гарантированно может получить 4000, называя 4-ки и 5-ки, пока Боб не поставит цифру на позицию , в этом случае Алиса выбирает нули до конца игры.
Либо Боб ставит 4-ку или 5-ку на позицию  и в таком случае она заканчивает игру одними 9-ками. Она должна обеспечивать  каждый раз, называя цифру 5, так как Боб может поставить эту 5 на позицию , а также она должна гарантировать, что  каждый раз, когда выбирает цифру 4, чтобы Боб не поставил 4 на позицию Она может сделать это следующим образом.
   Если в какой-то момент игры  и  одинаковы, Алиса называет либо 4, либо 5. В любой другой ситуации обозначим через u и v, соответственно, символы в  и , стоящие в самой левой позиции и отличные друг от друга. Если ( в таком случае ), Алиса называет 5, если ( в этом случае ), она называет 4. Никогда не может случиться, что  и  , а если  и  , оба неравенства   и  выполняются, так что Алиса может смело выбирать либо 4, либо 5.
    
   С другой стороны, Боб легко гарантирует себе 4000, если сразу же поставит 4 или меньшую цифру на позицию , либо 5 или большую цифру на позицию . Затем он не меняет первую звёздочку в другом числе, пока не появится не ноль (в первом случае) или не девять (во втором случае). Таким образом он получает либо , либо , а, может, и гораздо лучший результат.


   Эта задача уходит своими корнями, по меньшей мере, к 6-й Всесоюзной Математической олимпиаде в Челябинске  1972 года..


 ТРОЙНАЯ ДУЭЛЬ (THREE-WAY DUEL)
 
  Об этой бородатой задаче  мне напомнил  Ричард Плотц из Провиденса, штат Род-Айленд. (Dr. Richard Plotz of Providence, RI) Она появлялась во многих версиях, одну из которых можно проследить хотя бы до книги головоломок Хьюберта Филлипса “Вoпросы и ответы”   выпущенной в 1938 году лондонским издательством “J.M.Dent & Sons” (Hubert Phillips Question Time, published by J. M. Dent & Sons, London)


  Совершенно очевидно, что Алиса не должна целиться в Боба, если она попадёт, то на следующем шаге её подстрелит Кэрол, конец игры.
   В случае удачи с Кэрол Алисе предстоит дуэль с Бобом,  а Боб лучше попадает в цель, и стреляет он первым. Шансы выжить у неё определённо меньше ⅓.
  (В действительности, если мы обозначим p - вероятность Алисы остаться в живых, если Боб стреляет первым и q -- вероятность (большую) выживания Алисы, когда она стреляет первой, то получаем  и , отсюда  Не радостно для Алисы.)
  Если же она промахивается, то в Кэрол будет стрелять Боб. Если он выигрывает, то опять предстоит дуэль между Алисой и Бобом, но в этот раз первой стреляет Алиса, и её шансы выжить превышают ⅓ (а точнее, равны 3/7).
  Если Боб терпит неудачу, Кэрол застрелит его, и  у Алисы будет один шанс попасть в Кэрол. Вероятность выжить для неё в этом случае равна ровно ⅓.
  Суть в том, что независимо от того, попадёт ли Боб в Кэрол или нет, Алисе лучше промахнуться, чем попасть, стреляя в Кэрол. И намного лучше промахнуться, чем попасть, стреляя в Боба.
  Итак, лучшая стратегия для Алисы -- пожертвовать преимуществом первого хода и стрелять в воздух.
