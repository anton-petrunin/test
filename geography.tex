\chapter*{География(!)}
\addcontentsline{toc}{chapter}{География(!)}

\setlength{\epigraphwidth}{.45\textwidth}
\epigraph{Без географии мы были бы нигде.}{---Джимми Баффет (1946---)} 

Да, данная глава не принадлежит этой книге. %(не из этой книги).
Некоторые из задач, представленных здесь, конечно, математические по своей природе, %(mathematical in nature) 
но, в основном, они включены в книгу потому, что доставляют, как мне кажется, большое удовольствие любителям математических головоломок.
Мой издатель уверял меня, что без «Географии» стоимость книги была бы такой же.

Так что эта глава как бы бесплатное приложение, и её можно пропустить с чистой совестью.

Основная тема приведённых ниже задач --- поверхность планеты Земля.
Хотя предпочтение всё-таки отдаётся моей родине --- Соединённым Штатам Америки. 
О чём прошу прощения у читателей из других стран.
Я буду очень благодарен за подобные задачи про разные страны, присланные мне на pw@akpeters.com.

Несколько из этих задач показывают до какой степени %( test the degree) 
картографическая проекция %( на плоскость) (planar projections) 
искажает наше представление о земном шаре. %(глобусе, globe).
Вот одна из них: 

\subsection*{Африка}
\rindex{Африка}

Какой штат США ближе всего к Африке? 

\paragraph{Решение:} Штат Мэн.\heart
  
Он совсем не близко --- проверьте по глобусу.
Но если вы летите по ортодромии (в картографии и навигации ортодромия --- название кратчайшего расстояния между двумя точками на поверхности Земли) из Майами, скажем, в Касабланку, вначале ваш курс %(путь)
будет лежать на Северо-Восток вдоль восточного побережья и пройдёт очень близко к штату Мэн.
%(not missing Maine by much)

\medskip

Дальше попробуйте сами.

\subsection*{На восток от Рино}%EAST OF RENO 
\rindex{На восток от Рино}

Какой самый большой город в США к востоку от Рино, Невада и к западу от Денвера, Колорадо? 

\subsection*{Телефонный звонок}%(THE PHONE CALL) 
\rindex{Телефонный звонок}

Представьте, что вы звоните из какого-то штата Восточного побережья в один из штатов Западного побережья США, и на обеих концах одно и то же время суток.
Как такое может быть? 
  

\subsection*{Диаметр Соединённых штатов}%(THE DIAMETR OF US)  
\rindex{Диаметр Соединённых штатов}

В каких двух штатах находятся две самые удалённые точки США? 
  

\subsection*{На юг от Ки-Уэст}% ( SOUTH FROM KEY-WEST) 
\rindex{На юг от Ки-Уэст}

Если вы летите на юг от города Ки-Уэст, Флорида, какая южно-американская страна вам 
встретится первой? 

\subsection*{Индейцы на среднем западе}% (INDIANS IN THE MIDWEST) 
\rindex{Индейцы на среднем западе}

Среди штатов Среднего Запада США только один имеет название не индейского происхождения? Который?    

\subsection*{Самый большой второй по величине город}% (THE LARGEST SECOND-LARGEST CITY)  
\rindex{Самый большой второй по величине город}

Какой город в США самый большой среди городов с одинаковыми именами, но не больше города США с таким же именем?

\medskip

Эта формулировка может показаться несколько путаной.
Спросим по-другому: скажем, что город (в США) «находится в тени», если существует б\'{о}льший город с таким же именем.
Например, Портланд, штат Мэн, находится в тени города Портланд, штат Орегон.

Итак, наш вопрос прозвучит теперь так: какой наибольший город в США находится в тени?   

\subsection*{Естественные границы}% (THE NATURAL BORDERS) 
\rindex{Естественные границы}

Граница штата может быть естественной (определяться водоёмами, горами и пр.) или 
закреплённой в законе искусственной линией --- в одном знаменитом случае (связанном со штатом Делавэр и Пенсильванией ) --- это дуга окружности.
Три штата --- Колорадо, Юта и Вайоминг имеют только искусственные границы.
Какой штат обладает только естественными границами?   

\subsection*{Непересекаемые границы}% (THE UNCROSSABLE BORDER) 
\rindex{Непересекаемые границы}

Говоря о границах штатов, можете ли вы найти такой штат, который нельзя пересечь на автомобиле? Другими словами укажите два штата, имеющих общую границу, через которую, однако, невозможно напрямую проехать на автомобиле из одного штата в другой?   

\subsection*{Отдел странных названий}% (DEPARTMENT OF ODD NAMES) 
\rindex{Отдел странных названий}

Что особенного в неком местечке, именуемом Уэст-Куодди-Хед, %(West Quoddy Head) 
штат Мэн?   

\subsection*{Городской и деревенский}%(URBAN AND RURAL) 
\rindex{Городской и деревенский}

Данная задача скорее более социологического плана.
В наши дни большинство американцев --- порядка 75\% --- живут в так называемых «городских агломерациях (урбанизированных зонах)».
Перепись населения 2000 года относит к «городскому» 100\% населения одного из штатов и только 27,6\% населения другого штата, который отдалён от первого всего лишь на несколько сот миль.
Можете назвать эти два штата?   

\subsection*{Города на север и на юг}% (CITIES NORTH AND SOUTH) 
\rindex{Города на север и на юг}

Как обстоит дело с вашей визуализацией континентов?
Проверьте, как точно вы представляете себе %(визуаилизируете) 
карту мира.
Расставьте следующие четыре города по порядку с юга на север: 
Галифакс, Новая Шотландия; %(Halifax, Nova Scotia) 
Токио, Япония; %(Tokyo, Japan); 
Венеция, Италия; %(Venice, Italy); 
Алжир, Алжир. %(Algiers, Algeria).

\subsection*{Город в один слог}% (THE ONE-SYLLABLE CITY) 
\rindex{Город в один слог}

Какой город в США, имя которого состоит из одного слога, самый большой? 

\subsection*{Вашингтоны и феминисты}%(WASHINGTONS AND FEMINISTS) 
\rindex{Вашингтоны и феминисты}

Данная задача --- это своеобразный тест на знание не только карты штатов США, но и их английских названий.
Чтобы найти решение этой задачи, вам придётся пользоваться исключительно оригинальными английскими названиями штатов.
Итак, вопрос: 

Сможете ли вы проложить маршрут для автомобиля из города Сиэтл, штат Вашингтон, 
в Вашингтон, округ Колумбия, таким образом, чтобы 
названия всех штатов, через которые вы планируете проехать, начинались только с букв, составляющих слово «WOMAN»?


\medskip

Наша последняя географическая задача напоминает нам, что пора возвращаться к математике. %(signals us a(slight) return to mathematics) направляет 

\subsection*{Учёный и медведь}%(THE NATURALIST AND THE BEAR) 
\rindex{Учёный и медведь}

Учёная-биолог покинула лагерь экспедиции, прошла 10 миль на юг, затем 10 миль на восток и тут заметила и сфотографировала медведя.
Пройдя 10 миль на север, она пришла обратно в лагерь.

\medskip

Вы не видели фотографии, но всё равно знаете, какого цвета был медведь, не так ли?
