\chapter*{Послесловие}
\addcontentsline{toc}{chapter}{Послесловие}

\setlength{\epigraphwidth}{.6\textwidth}
\epigraph{Математика --- не церемониальный марш по гладкой дороге, а путешествие по незнакомой местности, где исследователи часто рискуют заблудиться.
Строгость должна стать указанием для историка о том, что данная местность нанесена на карту, а настоящие исследователи отправились дальше.}{---У. С. Энглин
%https://biography.wikireading.ru/57144
}

Книга, которую  вы прочитали, представляет собой не книгу по математике, a коллекцию математических головоломок, или, по крайней мере, она была так задумана.
Она посвящена задачам занимательным, а не важным.
Она не выстраивает теории, не вносит структуру и не навязывает правил, она также не требует продолжительного внимания.

Даже сторонники  подхода к математике, ориентированного на решение задач (такие, как Тим Гауэрс, автор статьи «Две культуры в математике»%
\footnote{Gowers, W. T. ``The two cultures of mathematics.'' \emph{Mathematics: frontiers and perspectives.} (2000) 65--78.; перевод на русский Никиты Калинина \href{http://www.mathcenter.spb.ru/nikaan/misc/Two_cultures.pdf}{\url{www.mathcenter.spb.ru/nikaan/misc/Two_cultures.pdf}}})
ужаснулись бы идее изучать математику по книжке головоломок.
Ваш автор и не возражает.

Но всё же, меня не оставляет чувство, что способность понять и оценить головоломки, даже с однотипными решениями, очень полезна.
Я не пытался здесь охватить  философию решения задач, как это сделал Пойа и другие, предпочтя дать задачам говорить самим за себя.
А задачи действительно говорят, и они говорят правду.

\begin{flushright}
---Питер Винклер
\\
9 июля 2003
\end{flushright}
