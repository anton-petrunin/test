\section*{Решения и комментарии}


Для проверки правильности ответов  бы можете воспользоваться атласом, глобусом, альманахом  или итогами переписи населения США 2000 года. 
Давайте посмотрим, насколько верны были ваши предположения...
%(догадки)


\subsubsection*{На восток от Рино}% (EAST OF RENO)


Вопрос о самом большом городе может оказаться довольно щекотливым; 
%(awkward question)  
стандартно  это понятие определяется количеством  населения (не площадью!) в официальных границах города, что, безусловно, может привести к неверным выводам
при наличии  городских агломераций (урбанизированных зон).
%(can be misleading with respect to metropolitan area).
Так, например, согласно данным альманаха город Джэксонвилл, штат Флорида, представляется больше, чем Атланта, штат Джорджия,  несмотря на то, что население всей городской агломерации Атланты превышает население Джэксонвилла почти в четыре раза.


Но в нашей задаче нам не понадобятся такие тонкости. Самый большой город к востоку от Рино  и к западу от Денвера, в любом случае, Лос-Анджелес, Калифорния.                                                                                                                                  \heart






\subsubsection*{Телефонный звонок}%ТЕЛЕФОННЫЙ ЗВОНОК (THE PHONE CALL)


«Восточное побережье»  Соединённых Штатов Америки включает в себя  восточные штаты, имеющие выход к Атлантическому океану,  от штата Мэн на севере до штата Флорида на юге. 
К «Западному  побережью»  относятся штаты Вашингтон, Орегон и Калифорния,
к  которым, если хотите, вы можете добавить Аляску и даже Гавайи,  но это не так уж важно в данном случае. %(but it doesn’t help).


Обычно, делая подобные звонки, мы имеем разницу во времени между Восточным и Западным побережьем в 3 часа.  Мы можем избавиться от одного часа, позвонив из  западного района так называемой  «ручки ковша» Флориды --- её северо-западной части, % ( Florida panhandle) 
скажем, из города Пенс\'акола, который находится в  Центральном часовом поясе.  
Чтобы избавиться ещё от одного часа, мы звоним в один из городов самого  восточного района  штата Орегон (скажем, Онтарио),  в котором Горное время. 
Оставшийся час исчезнет, если звонок будет сделан  из Пенс\'аколы между двумя и тремя часами ночи, при переходе с Летнего времени на Зимнее. 
В этот момент в Центральном часовом поясе время уже переведут на один  час назад, а Горное время все ещё  будет прежним.\heart   


                            
                                                                                                                                     
\subsubsection*{Диаметр Соединённых штатов}%ДИАМЕТР СОЕДИНЁННЫХ ШТАТОВ (THE DIAMETR OF US) 


Очевидно, это либо Гавайи и Мэн, либо Аляска и Флорида. Или это Гавайи и Аляска?


Удивительным образом, ни одно, ни другое, ни третье.  Правильный ответ --- Гавайи и Флорида.\heart




\subsubsection*{На юг от Ки-Уэст}%НА ЮГ ОТ КИ-УЭСТ ( SOUTH FROM KEY-WEST)


Это, без сомнения, каверзный  вопрос. %(a trick question).  
Вы не пересечёте ни одну страну Южной Америки. 
Ваш путь пройдет к западу от континента. 
(Ваш путь пройдет вдоль западного побережья всего континента.)\heart


\subsubsection*{Индейцы на среднем западе}%ИНДЕЙЦЫ НА СРЕДНЕМ ЗАПАДЕ (INDIANS IN THE MIDWEST)


По определению к штатам  Среднего Запада относятся Миннесота, Висконсин, Айова,
Иллинойс, Миссури, Мичиган, Огайо, Канзас и Небраска --- все названия индейского
происхождения, и остается ответ --- Индиана!\heart


Любопытно, что только один штат к востоку от Миссисипи имеет столицу, чье имя индейского происхождения --- Флорида (Таллахасси).




\subsubsection*{Самый большой второй по ввеличине город}%САМЫЙ БОЛЬШОЙ ВТОРОЙ ПО ВЕЛИЧИНЕ ГОРОД (THE LARGEST SECOND-LARGEST CITY)     


Портланд, штат Мэн, Спрингфилд, что-то?  Популярные предположения, но не верные.
До 1975 года или около того, правильный ответ был бы Канзас-Сити, штат Канзас, который затеняется городом Канзас-Сити, Миссури.
Затем некоторое время победителем был Колумбус, Джорджия, находящийся в тени столицы Огайо. Однако, мы живем в эпоху пригородов, %(in the age of suburbia), 
перепись населения 2000 года показывает, что теперь городу Глендейл, штат Калифорния (находится в тени Глендейла, Аризона) принадлежит эта сомнительная %(obscure) 
честь.\heart




\subsubsection*{Естественные границы}%ЕСТЕСТВЕННЫЕ ГРАНИЦЫ (THE NATURAL BORDERS)


Конечно, Гавайи  имеют только естественные границы. Возможно, вы подумали, что это было слишком легко, но люди часто не видят, что находится у них под носом. 
%(не замечают очевидного)  %(folks often have a blind spot).
\heart




\subsubsection*{Непересекаемые границы}%НЕПЕРЕСЕКАЕМЫЕ ГРАНИЦЫ  (THE UNCROSSABLE BORDER)


Здесь намного сложнее. Висконсин и Мичиган имеют общую длинную границу по озеру Мичиган, но вы можете пересечь её на пароме Манитовок-Ладингтон, сидя в своей машине. Паром из Монток Пойнт %(Mantauk Point) 
(штат Нью-Йорк) на остров Блок %(Block Island) 
(штат Род-Айленд) пересекает не очень хорошо известную границу между этими двумя штатами, и он только для пассажиров. 
Возможно, существуют и другие решения данной задачи.\heart
                                                                                                                    


Можно задать схожий вопрос о части штата, попасть в которую на автомобиле из остального штата возможно только, проехав через другой штат (или Канаду, в случае с 
Пойнт Робертс. %(Point Roberts). 
Существует несколько таких мест, особенно около вечно изменчивой реки Миссисипи.




\subsubsection*{Отдел странных названий}%ОТДЕЛ СТРАННЫХ НАЗВАНИЙ (DEPARTMENT OF ODD NAMES)


Уэст-Куодди-Хед %(West Quoddy Head) 
--- самая восточная точка континентальных штатов США.\heart
                                                                                                                


Иногда вам может встретиться утверждение,  что, если опустить  требование «континентальный», мыс Врангеля %(Cape Wrangel) 
на острове Атту, %(Attu Island) 
штат Аляска, является самой восточной точкой США,  но я не принимаю в рассчет эти «Гринвич-централизованные»  доводы. 
%(but I do not buy Greenwich-centered reasoning) 
Назовёте ли вы мыс Врангеля самой восточной точкой Аляски?




\subsubsection*{Городской и деревенский}%ГОРОДСКОЙ И ДЕРЕВЕНСКИЙ (URBAN AND RURAL)


Нью-Джерси и Вермонт. Эту и множество другой интересной информации вы можете найти на сайте:\\
\texttt{http//www.census.gov/prod/2002pubs/01statab/pop.pdf} 
 \heart                                                                                                      




\subsubsection*{Города на север и на юг}%ГОРОДА НА СЕВЕР И НА ЮГ (CITIES NORTH AND SOUTH)


Токио, Алжир, Галифакс и, наконец, Венеция. Широты, соответственно
$35^\circ 40’$ с.ш.;  $36^\circ 50’$ с.ш.; $44^\circ 53’$ с.ш.; и $45^\circ 26’$ с.ш. 
Обратите внимание --- последние два города разделяет 45-я параллель, и это позволяет нам яснее увидеть, что Венеция находится севернее. Один уроженец Новой Шотландии однажды проспорил мне по этому случаю 1 доллар.\heart




\subsubsection*{Город в один слог}%ГОРОД В ОДИН СЛОГ (THE ONE-SYLLABLE CITY)


Йорк, штат Пенсильвания, и Трой, штат Нью-Йорк, называются чаще всего, но все же Флинт, штат Мичиган, несмотря на существенное сокращение населения за последние годы, остаётся единственным однослоговым городом в США с населением более 100 тыс. человек.  Хотя, если судить по тому, как произносят названия городов местные жители, победителем, несомненно, будет Нью-Арк («Норк», с длинным «о»),  штат Нью-Джерси. \heart






\subsubsection*{Вашингтоны и феминисты}%ВАШИНГТОНЫ И ФЕМИНИСТЫ (WASHINGTONS AND FEMINISTS)


Без проблем. Вначале ваш маршрут пойдет на юг ---  через Орегон (Oregon), 
Неваду (Nevada) 
и Аризону, (Arizona),  
затем на восток сквозь Нью-Мексико (New Mexico) 
в  Оклахомовскую «ручку ковша»,  из северо-восточного  угла Оклахомы (Oklahoma) 
вы попадаете в  Миссури (Missouri). 
Здесь вам надо будет повернуть на север и из северо-западного угла штата проехать в Небраску (Nebraska),  продолжая путь на  запад в Вайоминг (Wyoming)  и на север в Монтану (Montana) --- довольно большой круг для того, чтобы объехать Айдахо (Idaho). 
В конце концов, вы сможете снова развернуться и поехать на восток через Северную Дакоту (North Dakota), 
Миннесоту (Minnesota), 
Висконсин (Wiskonsin) 
и Мичиган (Michigan).
Теперь берите курс на юг в Огайо (Ohio) и на восток сквозь Западную Виргинию (West Virginia) в Мериленд (Maryland),
 и в Вашингтон (Washington DC), округ Колумбия.\heart


Чтобы пройти по этому маршруту, вам придется несколько раз (или на долгое время) покинуть национальную систему межштатных автомагистралей, но мы предполагаем,  вы никуда не торопитесь.


\subsubsection*{Учёный и медведь}%УЧЕНЫЙ И МЕДВЕДЬ  (THE NATURALIST AND THE BEAR)




Начальная идея была, конечно, что лагерь экспедиции находился на Северном полюсе, так как маршрут ученой (10 миль на юг, 10 миль на восток и 10 миль на север)  является замкнутым контуром, %(to have been a closed loop),  
следовательно  медведь был белый.


Однако, как было замечено в одной из рубрик %(колонок) 
Мартина Гарднера, %(Martin Gardner’s columns), 
на поверхности Земли существует бесконечно много других точек, где подобный путь будет замкнутым.


Некоторые из этих точек лежат на окружности  с центром в Южном полюсе и радиусом немного меньше $10 + 5/\pi$ миль.
Начав   прогулку с такой точки,  наша учёная после первых 10 миль окажется в некой точке $P$,  
находящейся на расстоянии чуть меньше $5/\pi$ миль от Южного полюса.  
Повернув на Восток и пройдя 10 миль, она обогнёт весь мир и вернётся в точку $P$, откуда 10 миль на Север приведут её обратно в лагерь.
Другая окружность, радиусом чуть меньше $10 +  5/\pi$ миль тоже сработает, во второй (Восточной) части пути наша учёная должна будет 2 раза обойти вокруг Южного полюса  и так далее.
В Антарктике медведи не водятся, но если бы водились, то, наверное, были бы белыми.
Так что ответ задачи не изменится.\heart