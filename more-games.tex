 
“Half this game is 90\% mental.”
-- Danny Ozark manager of the Phillies baseball team.


Половина этой игры на 90\% происходит в голове.
   Дэнни Озарк, тренер Филадельфийской бейсбольной команды.




Анализ игры часто требует, в сущности, решения двух задач: нахождения правильной стратегии и нахождения убедительного аргумента (или правильной стратегии для второго игрока), который показывает, что первая стратегия -- наилучшая из возможных.
   Хотя, иногда можно отделаться довольно легко. Рассмотрим следующую невинную с виду задачу.


ЩЁЛК (CHOMP)


Два игрока по очереди откусывают от прямоугольной шоколадки, сотоящей из m x n  квадратных долек. Каждый раз игрок выбирает дольку и откусывает её вместе с оставшимися сверху и/или справа от неё дольками. Каждый игрок старается избежать левую нижнюю дольку, которая отравлена.
     Докажите, что если шоколадка состоит больше, чем из одной дольки, тогда у первого игрока есть выигрышная стратегия.


                        Рис. Стр 95


РЕШЕНИЕ: Либо у первого игрока (Алиса), либо у второго (Боб) должна быть выигрышная стратегия. Предположим, она будет у Боба. Тогда, в частности, у Боба должен быть выигрышный ответ на первый ход Алисы, когда она просто откусывает верхнюю дольку справа. 
   Но какой бы ни был ответ Боба, Алиса может сделать таким же свой первый ход, что противоречит предположению о том, что Боб всегда может выиграть. Отсюда следует,
что выигрышная стратегия должна быть у Алисы.


   Такой вид доказательства известен как способ заимствованной стратегии и он, к сожалению, не говорит нам о том, как же, собственно, Алиса выигрывает игру.
В последней главе подробнее рассказывается об игре “Щёлк”, её истории и более общей версии. 


  Для решения оставшихся задач-игр используются разнообразные методы.




ДЕТЕРМИНИРОВАННЫЙ ПОКЕР (DETERMENISTIC POKER)


Не желая зависеть от капризного случая, Алиса и Боб решили сыграть в абсолютно детерминированную версию пятикарточного покера (т.н. дро-покер). Колода карт раскладывается на столе в открытую. Алиса выбирает 5 карт, затем Боб берёт 5 карт.
Алиса меняет любое число карт, заменённые карты выходят из игры, то же проделывает и Боб. Все действия производятся в открытую на виду  у оппонента. Игрок с лучшей комбинацией выигрывает. Поскольку Алиса ходит первая, то, если конечные комбинации оказываются равносильными, Боб объявляется победителем. Кто победит при оптимальной игре?


   Детерминированный покер -- это игра с полной информацией. В играх, содержащих скрытую информацию или одновременные ходы, может потребоваться вероятностная стратегия. Говорят, что набор таких стратегий (один для каждого игрока) нaходится в равновесии, если никто из игроков не может увеличить выигрыш, изменив свою стратегию, если остальные участники своих стратегий не меняют. Например, в игре “Камень, ножницы, бумага”  для достижения (уникального) равновесия стратегий необходимо, чтобы все игроки все три возможных варианта выбирали с одинаковой вероятностью.


 ШВЕДСКАЯ ЛОТЕРЕЯ (SWEDISH LOTTERY)




Для Шведской национальной лотереи предлагался следующий механизм игры: каждый участник выбирает  целое положительное число. Тот, кто называет наименьшее число, никем другим не выбранное, объявляется победителем. (Если нет числа, выбранного только одним участником, победителя нет.)
   Если в игре участвуют только три человека, и каждый применяет оптимальную, равновесную, вероятностную стратегию, чему равно наибольшее положительное число,  вероятность быть выбранным которого больше нуля?


БЛИНЫ (PANCAKES)


Алиса и Боб опять проголодались, и перед ними лежат две стопки блинов, высотой в m и n блинов. Каждый игрок по очереди должен съесть из большей стопки число блинов (ненулевое), кратное количеству блинов в меньшей стопке. Разумеется, последний блин в каждой стопке непропечённый, так что игрок, который первым заканчивает стопку, проигрывает.


   Для какой пары (m,n) у Алисы (она ходит первая) имеется выигрышная стратегия?


  Что случается, если у игры обратная цель, то есть побеждает тот, кто первый закончит стопку?




  ОПРЕДЕЛЕНИЕ РАЗНОСТИ (DETERMINING A DIFFERENCE)


После завтрака Алиса и Боб решают отдохнуть и сыграть в простую числовую игру. Алиса выбирает цифру и Боб подставляет её вместо звёздочки в выражение “ ” Алиса пытается сделать окончательную разность максимальной, Боб - минимальной.  Какая разность получится при оптимальной игре?




ТРОЙНАЯ ДУЭЛЬ (THREE-WAY DUEL)


Алиса, Боб и Кэрол устраивают тройную дуэль. Алиса -- плохой стрелок, попадает в цель, в среднем, только в  ⅓  случаев. Боб лучше, попадает в цель с вероятностью ⅔. Кэрол -- меткий стрелок, бьёт без промаха.
    
   Они стреляют по очереди, первой Алиса, вторым Боб и третьей  Кэрол, затем опять Алиса, и так далее, пока не останется только один стрелок. Каков для Алисы наилучший план действий ?
