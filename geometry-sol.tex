\section*{Решения и комментарии}

\subsubsection*{Окружности в пространстве}% (CIRCLES IN SPACE)

Да.
Построим на плоскости XY окружности радиуса 1, центр кoторых лежит на оси Х в точках 1 mod 4 (это будут точки …, (-7,0), (-3,0), (1,0), (5,0), (9,0), … ).
Обратите внимание, что каждая сфера с центром в начале координат пересекает эти окружности ровно в двух точках.
Остаток каждой такой сферы -- это объединение окружностей.

Рис. Стр 48 

Существуют и другие способы доказательства этого факта, например, с помощью торов, но я не знаю подхода более простого и элегантного, чем приведённый выше.

Эту милую задачу на разбиения я впервые услышал от Ника Пиппенгера (Nick Pippenger), профессора информатики Принстонского университета.

\subsubsection*{Магия кубов}% (MAGIC WITH CUBES)

Это можно сделать.
Для того, чтобы протащить единичный куб сквозь отверстие в другом единичном кубе, достаточно найти поперечное сечение (второго) куба, содержащее внутри себя единичный квадрат.
Тогда во втором кубе можно проделать сквозное квадратное отверстие (square cylindrical) со стороной чуть больше единицы, чтобы можно было протащить первый куб.
Можно сделать то же самое, с меньшим допуском, если второй куб только слегка меньше первого.
Самое простое (но не единственное) сечение, которое вы можете использовать -- правильный шестиугольник.
Он получается при рассечении куба через середины трёх рёбер и центр.
Можно увидеть этот шестиугольник, если смотреть на куб так, что одна из вершин совпадает с центром.

Пусть А -- проекция одной из видимых граней на плоскость.
Мы видим, что её большая диагональ имеет такую же длину (, как и диагональ единичного куба, так как that line has not been foreshortened.
Если мы сдвинем (slide) A к центру шестиугольника, затем растянем её до единичного куба B, растянутые углы B не достанут до вершин шестиугольника (так так расстояние между противоположными вершинами шестиугольника превышает расстояние между противоположными сторонами).

Рис. Стр 49

Из этого следует, что если мы слегка наклоним B, все четыре его угла будут лежать строго внутри шестиугольника.
Об этой очаровательной задаче, появлявшейся в колонке Мартина Гарднера, мне напомнил Григорий Галперин из университета Восточного Иллинойса.

\subsubsection*{Красные точки и синие точки}% (RED POINTS AND BLUE POINTS)

Среди всех возможных разбиений, возьмём то, при котором общая длина всех n отрезков минимальна.
Мы утверждаем, что такое разбиение не будет иметь пересечений.
Поскольку, если отрезок uv пересекает отрезок xy, то эти отрезки являются диагоналями выпуклого четырёхугольника uvxy, и, используя неравенство треугольника, мы видим, что при разбиении, содержащим стороны uy и xv, общая длина отрезков стала бы меньше.

Технику, которой мы здесь воспользовались, состоящую в нахождении нужного объекта, минимизируя или максимизируя некую величину, иногда называют вариационным методом, и он, как знают многие читатели, чрезвычайно полезен.
Следующая задача предлагает ещё один пример его применения.
Источник : Олимпиада Патнема 1960-х.

\subsubsection*{Прямая через две точки}% (LINE THROUGH TWO POINTS)

Эта знаменитая задача была гипотезой Сильвестра (Sylvester), сформулированной в 1893 году.
Впервые её доказал Тибор Галлаи (Tibor Gallai).
Но доказательство, приведённое ниже, придуманное в 1948 году Л.М.Келли ( L.M. Kelly)(American Mathematical Monthly, Vol.55) часто упоминалось Полем Эрдёшем (Paul Erdos), как пример доказательства из “Книги”.

Предположим, что каждая прямая, проходящая через две или более точки из множества X содержит, на самом деле, по меньшей мере, три точки из X.
Идея состоит в том, чтобы найти такую прямую L и такую точку P, не лежащую на L, чтобы расстояние от P до L было минимальным.
Поскольку L содержит, по меньшей мере, три точки из X, две из них, скажем, Q и R, лежат с одной стороны от перпендикуляра, опущенного с точки P на прямую L.
Но тогда,
если R дальняя точка, то Q находится ближе к прямой PR, чем P к L -- противоречие.

Рис. Стр 50

\subsubsection*{Пары на максимальном расстоянии}% (PAIRS AT MAXIMUM DISTANCE)

Для решения данной задачи из олимпиады Патнема 1957 года, будет полезно следующее наблюдение: если A,B и C,D -- две “максимальные пары” (пары точек из множества X, расстояние между которыми равно d), тогда отрезки AB и CD должны пересекаться (иначе одна из диагоналей прямоугольника ABCD будет иметь длину, превышающую d).

Предположим, что утверждение задачи неверно, и пусть наименьший контрпример имеет размер n.
Поскольку максимальных пар больше, чем n, и каждая состоит из двух точек, то должна существовать такая точка P, которая принадлежит трём максимальным парам (пусть это будут пары с точками A, B, C).
Каждые два из отрезков PA, PB и PC должны в точке P образовывать угол максимум в 60, и один из них, скажем, B, должен лежать между двумя другими.


Но тогда точке B будет довольно сложно образовать максимальную пару с какой-либо другой точкой, так как, если PQ была бы максимальной парой, то отрезок PQ должен был бы пересекать и PA, и PC, что невозможно.
Таким образом, можно совсем выбросить P из множества X, теряя при этом только одну максимальную пару и получая меньший контрпример.
Это противоречие завершает доказательство.

\subsubsection*{Монах на горе}% (MONK ON A MOUNTAIN)

Пожалуй, самое лёгкое решение -- это представить себе, что у монаха есть близнец, которому даны указания взобраться на гору во вторник утром точно тем же путём, каким шёл монах в понедельник.
Тогда монах должен встретить близнеца по дороге вниз, или, если они идут разными тропами, оказаться в какой-то момент на одной с ним высоте.

(Возможно, эта задача показалась вам слишком лёгкой, не волнуйтесь, гораздо более сложная её версия ожидает вас в Главе 11 -- “Крепкие орешки” [“Toughies”])

На эту древнюю задачу можно смотреть как на пример применения очень полезной теоремы о промежуточном значении, в которой говорится, что непрерывная функция должна принять все промежуточные значения.
В нашем случае функцией можно выбрать разность между высотой, на которой оказался монах в определённое время дня в понедельник, и высотой, на которой он был в то же самое время дня во вторник.
Начальное значение функции будет отрицательным (примерно минус высота горы Фудзияма), а конечное значение -- положительным, таким образом, в некой точке функция должна быть равна нулю.

Геометрически, вы можете представить, что высота, на которой находился монах в каждый из дней, представлена графиком, и два графика наложены друг на друга.
Должна существовать точка (или точки), где они пересекаются.

Задача о том, как вписать озеро Мичиган в квадрат, и о том, как разрезать сэндвич (плоскостью) так, чтобы разделить ветчину, сыр и хлеб точно пополам --- другие известные примеры применения теоремы о промежуточном значении.


\subsubsection*{Раскраска многогранника}% (PAINTING THE POLYHEDRON)

Предположим, что сфера вписана в P, триангулируем грани P, используя точки касания сферы.
Тогда треугольники по обе стороны любого ребра конгруэнтны, и, значит, имеют одинаковую площадь.
В каждой такой паре не более одного красного треугольника.
Из этого следует, что площадь красных граней самое большее равна площади зелёных, что противоречит условию задачи.

Рис. Стр 52

Эта задача пришла ко мне от Эмины Солянин (Emina Soljanin) из лабораторий Белла.
На иллюстрации представлена двумерная версия, где стороны и вершины многоугольника заменяют грани и рёбра многогранника P.

\subsubsection*{Круглые тени}% (CIRCULAR SHADOWS)

Эта, вполне могущая вас расстроить, задача, пришла к нам из 5-й Всесоюзной Математической Олимпиады в Риге 1971 года.
Простой способ, что заставит вас всё же напрячь интуицию, -- это взять плоскость, одновременно перпендикулярную двум плоскостям проекций, и вторую, параллельную ей, с другой стороны от тела, и начать сдвигать их.
В тот момент, когда они коснутся тела, они пройдут через противолежащие точки каждой из проекций, и расстояние между параллельными плоскостями будет равно общему диаметру проекций.


\subsubsection*{Полоски на плоскости}% (STRIPS IN THE PLANE) 

Как и предыдущая, данная задача, версия которой появлялась в ранних олимпиадах Патнема, представляет собой ещё один пример “интуитивно очевидного” факта, который нужно доказать.
Поскольку сравнивать бесконечные величины сложно, имеет смысл сосредоточиться на какой-нибудь конечной части плоскости.
Мы не можем контролировать углы между полосками, так что логично будет рассмотреть круг D радиуса r.
Предположим, полоски имеют ширину w1, w2,..., которые в сумме дают 1.
Получается, что они не могут покрыть даже круг D, в случае, если r=1.
Пересечение D с полоской шириной w лежит в прямоугольнике шириной w и длиной 2, и, следовательно, его площадь меньше 2w.
Таким образом, площадь, покрытaя полосками внутри D, меньше 2, а площадь D, конечно же, 

Доказательство выше указывает на то, что суммарная ширина полосок должна быть больше , чтобы покрыть единичный круг, но на самом деле, нельзя этого сделать, если только сумма не равна, по меньшей мере, 2 ( в этом случае параллельные полоски дают решение).
Существует очень милое доказательство этого утверждения.
Идея в том, чтобы продолжить задачу в 3-мерное пространство, взяв за D сечение единичного шара, проходящее через его центр.
Предположим, круг покрыт полосками, суммарная ширина которых W, и пусть S -- одна из полосок шириной, скажем, w.
Можно предположить, что либо оба края полоски пересекают D, либо один край пересекает и один касается.
Проектируя S вверх и вниз на поверхность шара, мы получаем пояс (или шапочку),
окружающую шар --- чья площадь, здесь вы можете показать себя в анализе, равна 
, независимо от положения полоски!
А поскольку площадь поверхности шара -- , нужно W 2, чтобы его покрыть.
А если вы не покрываете поверхность шара, вы не покрываете круг.

\subsubsection*{Ромбики в шестиугольнике}% (DIAMONDS IN HEXAGON)


Рис.стр 53

Доказательства без слов стали очень популярной темой в двух журналах Математической aссоциации Америки, Mathematics Magazine и The College Mathematics Journal.
Вы также можете найти эти задачи в книгах Роджера Б.
Нелсена Доказательства без слов и Доказательства без слов II (Proofs Without Words and Proofs Without Words II, by Roger B Nelsen), изданными МАА.
Задача о “ромбиках” в шестиугольнике появляется в первом томе как “Задача o калиссонах” (The Problem of the Calissons).

\subsubsection*{Замощение ромба}% (RHOMBUS TILING)

Пусть u -- одна из сторон 2n-угольника, а u-ромб -- любой из n-1 ромбов, у которых u является одним из двух векторов.
В замощении, плитка, прилегающая к u-стороне должна быть u-ромбом, как и плитка с другой стороны от этого ромба, и так далее, пока мы не достигнем противоположной стороны 2n-угольника.
Заметьте, что каждый шаг на этом пути делается в одном направлении (а именно, вправо или влево) относительно вектора u, также должен идти и любой другой путь u-ромба.
Но тогда не может быть других u-ромбов, поскольку они сгенерируют пути, которым негде будет закончиться.

Рис. Стр 54


Аналогично определяемый путь для другой стороны v должен пересечь u, и общая плитка, разумеется, состоит из u и v.
Могут ли они пересечься дважды? Нет, потому что при втором пересечении угол между u и v был бы больше внутри общего ромба.

Данная задача досталась мне от Дейны Рэндaлл из Технологическoго института Джорджии.
(Dana Randall of Georgia Tech).

\subsubsection*{Векторы на многограннике}% (VECTORS ON A POLYHEDRON)

На эту задачу обратил моё внимание Ювал Перес, профессор факультета статистистики Калифорнийского университета в Беркли ( (Yuval Peres of the Department of Statistics at UC Berkeley).
Самый простой способ понять, что сумма векторов должна быть нулевой, это провести следующий умственный эксперимент.
Накачаем воздух в (жёсткий) многогранник и отметим, что давление на грань есть сила, действующая по нормали и её величина пропорциональна площади поверхности грани.
Давление на грани должно быть уравновешено, иначе многогранник бы двигался по собственному усмотрению.

\subsubsection*{Три окружности}% (THREE CIRCLES)

Эта задача является лучшим известным мне примером эффективности повышения размерности пространства.
Заменим каждую окружность сферой, чьё пересечение с плоскостью и есть заданная окружность.
Теперь, каждой паре сфер соответствует конус, и искомые точки являются вершинами трёх конусов.
Но все эти вершины лежат на плоскости, которая касается сфер сверху, и точно так
же, они все лежат на плоскости, которая касается сфер снизу.
Отсюда следует, 
oни принадлежат пересечению двух плоскостей -- прямой! 

Мне кажется, это старинная классическая задача.
Впервые я услышал её от Дейны Рэндaлл (Dana Randall), из колледжа компьютерных наук Технологическoго института Джорджии.
Вадим Жарницкий, из университета Иллинойса, заметил, что можно задать аналогичный вопрос о четырёх сферах в 3-мерном пространстве: будут ли вершины шести конусов, определяемых данными сферами, лежать на одной плоскости? И действительно, oни будут, и один из способов это доказать -- повысить размерность до 4-х.

\subsubsection*{Сфера и четырёхугольник}% (SPHERE AND QUADRIATERAL)

Эту задачу я получил от Тани Ховановой, приглашенного научного сотрудника Принстонского университета в рамках программы по прикладной и вычислительной математике.
У неё есть коллекция задач, которые она зовёт “гробами”.
Oна пишет:

“Математико-механический факультет Московского Государственного университета, самая престижная математическая школа России, в своё время (1975) очень активно пытался препятствовать поступлению еврейских (и других “нежелательных”) студентов на факультет.
Один из методов, используемых в этих целях, был таков: неугодным студентам давался на устном экзамене отдельный набор задач.
Задачи эти выбирались очень аккуратно, они имели простое решение (чтобы факультет мог избежать скандала), которое было почти невозможно найти.
Любому, кто проваливал задачу, могли с лёгкостью отказать в приёме, так что подобная система эффективно контролировала поступление на МехМат.
Такого сорта задачи неформально называли “гробами”. 

Следующее решение, действительно, найти трудно, но, полагаю, не совсем невозможно, если вы чётко понимаете, что лучший способ доказать принадлежность четырёх точек одной плоскости -- это найти точку, лежащую на прямых, проходящих через две пары точек (пары общих точек не имеют).
Для начала отметим, что каждая вершина четырёхугольника i находится на одинаковом расстоянии di до точек касания образующих её сторон.
Пусть 1/di -- масса вершины i, тогда центр масс двух соседних вершин -- точка касания их общей стороны.
Из этого следует, что центр масс всех четырёх точек лежит на прямой, соединяющей противоположные точки касания, и,
значит, это и есть искомая точка.

\subsubsection*{Восьмёрки на плоскости}% (FIGURES 8S IN THE PLANE)

Эта задача известна уже около 50-ти лет.
Мне говорили, что автором её является великий тополог, профессор университета штата Техас Роберт Ли Мур (Robert Lee Moore; 1882—1974).
Читатели, незнакомые с разными степенями “бесконечности”, могут уже пребывать в недоумении: ведь очевидно, что на плоскости можно нарисовать бесконечно много восьмёрок, например, поместив по одной восьмёрке внутри каждой клетки квадратной решётки.
О таком множестве мы говорим, что оно счётно, что означает, восьмёрки можно пронумеровать натуральными числами, так, что каждой восьмёрке будет соответствовать только одно число.

Множество целых чисел, множество всех пар целых чисел, и, таким образом, множество рациональных чисел -- всё это счётные множества, но, как было замечено блестящим (хотя и часто депрессивным) математиком Георгом Кантором в 1878 году, множество вещественных чисел не является счётным.
Можно было бы начертить на плоскости концетрические окружности всех возможных положительных вещественных диаметров, и, следовательно, если бы в задаче говорилось об окружностях вместо восьмёрок, ответ был бы “несчётно много”, или, более точно, “ равномощно множеству вещественных чисел” (the cardinality of the reals).
Тем не менее, мы можем нарисовать только счётное число восьмёрок.
Каждой восьмёрке поставим в соответствие пару рациональных точек (точки на плоскости, чьи обе координаты -- рациональные числа), по одному числу каждой петле.
Восьмёрки не могут иметь общих пар точек.
Соответственно, мощность такого множества восьмёрок не больше, чем мощность множества пар из пар рациональных чисел, которое является счётным.

Более хитрую версию данной задачи смотрите в Главе 11 (“Крепкие орешки”).
